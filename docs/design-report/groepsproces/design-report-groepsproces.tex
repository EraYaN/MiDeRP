\documentclass{report}
% Include all project wide packages here.
\usepackage{fullpage}
\usepackage[style=ieee]{biblatex}
\usepackage[dutch]{babel}
\usepackage[T1]{fontenc}
\usepackage{titlesec, blindtext, color}
\definecolor{gray75}{gray}{0.75}
\newcommand{\hsp}{\hspace{20pt}}
\titleformat{\chapter}[hang]{\Huge\bfseries}{\thechapter\hsp\textcolor{gray75}{|}\hsp}{0pt}{\Huge\bfseries}
\renewcommand{\familydefault}{\sfdefault}
\usepackage[math]{iwona}

\addbibresource{../../library/bibliography.bib}

\title{EPO-2: Mid-term Design Report - Reflectie op het groepsproces}
\author{Luc Does}

\begin{document}
\chapter{Reflectie op het groepsproces}
Bij de start van een project komt het vaak voor dat de groepsleden elkaar nog moeten leren kennen om efficiënt te kunnen werken. Bij het EPO-1 project was hier in mindere mate sprake van aangezien de meeste projectleden elkaar hadden ontmoet bij het EOW en de groepen toen zelf gevormd zijn. Bij EPO-2 was hier geen sprake meer van en kenden de projectleden maar enkele of geen andere mensen in het team.
\newline

Ook in onze groep waren slechts enkelen met elkaar bekend. Dit zorgde ervoor dat de communicatie bij het begin van het project wat stroef van start ging. Gelukkig was dit stadium snel voorbij en hadden wij de afspraken gemaakt waarop het groepswerk gebaseerd zou worden.
\newline

Bij het starten van een project moeten er altijd afspraken gemaakt worden. Wanneer dit niet gebeurt zijn er zaken onduidelijk en daarover kan de voortgang onder lijden. De afspraken die wij hadden gemaakt bevatten bijvoorbeeld het gebruik van \LaTeX, Github en de begin- en eindtijden van de projectmiddagen. Ook was er in de derde projectweek een standaardsjabloon voor documenten in \LaTeX. Enkele van deze afspraken gaven nog wel een aantal problemen, Github synchronisaties verliepen niet vloeiend en \LaTeX  bleek ook nog de nodige ervaring nodig te hebben. Tot ons geluk  konden de mensen die al ervaring hadden met Github en \LaTeX  verheldering brengen in de wijze waarop deze hulpstukken gebruikt kunnen worden.
\newline

Het is  vanzelfsprekend dat iedereen zijn best doet om zich aan de afspraken te houden, maar dan moeten die afspraken wel bekend zijn. Aangezien bijna alle afspraken op de eerste projectdag gemaakt en Joris die dag niet aanwezig was waren en nog enkele dingen onduidelijk. Dit had tot gevolg de groep tijd verloor aan ondermaats werk en het uitleggen van de afspraken. Na een middag in gesprek hebben we Joris aangespoord zich aan de afspraken te houden en aldus waren de problemen grotendeels verholpen.
\newline

Een belangrijk punt in het groepsproces was de werkverdeling. In de weken van het derde kwartaal deelden we de groep op in drie groepen van elk twee personen. Door deze werkwijze konden wij verschillende implementaties maken voor hetzelfde probleem. Dit stelde ons in staat om allemaal dezelfde dingen te leren terwijl iedereen oplossingen kon bedenken die handig waren voor de uiteindelijke implementatie. Een nadeel hiervan was dat de vordering in de opdrachten niet vlug ging, en dat de subgroepen uit elkaar gingen lopen in opzichte van niveau en werksnelheid.
\newline

Dit heeft ertoe geleid dat de meeste gebruikte implementaties afkomstig zijn van Erwin en Robin, iets wat kwalitatief gezien een goede keus is maar wat wel voor problemen zorgt voor de rest van de groep. Deze moeten namelijk extra energie steken in het begrijpen van een implementatie die niet door zichzelf gemaakt zijn.
\newline

In het vierde kwartaal werden er geen subgroepen meer aangemaakt die aan dezelfde opdrachten werkte, maar werd de groep opgedeeld in subgroepen die individuele onderdelen behandelden. Erwin hield zich bezig met de VHDL van de robot, Robin met de routeplanner en de Zigbee. Samen hielden zij zich ook bezig met het Seven-segment-display. Luc, Tijmen en Chy namen de sensor onder handen.
\newline

Maar het werk aan de robot was niet de enige zorg in het vierde kwartaal. Het design-report moest ook worden gemaakt. Hieraan heeft iedereen een eigen bijdrage geleverd. Overigens was het werk aan de robot nog niet af toen het werk aan het design-report begon. Vandaar dat Erwin en Robin zich nog focuste op de robot en Chy, Joris, Tijmen en Luc zich verdiepten in het uitwerken van het verslag.
\newline

Al met al verliep het groepsproces niet vlekkeloos, maar we hebben geleerd als een groep te functioneren op een effectieve en voor ons plezierige manier. Deze werkwijze heeft ertoe geleid dat we een werkend product hebben opgeleverd welke als een van de eerste compleet was.


\end{document}