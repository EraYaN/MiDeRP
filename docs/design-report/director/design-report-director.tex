\documentclass{report}
% Include all project wide packages here.
\usepackage{fullpage}
\usepackage[style=ieee]{biblatex}
\usepackage[dutch]{babel}
\usepackage[T1]{fontenc}
\usepackage{titlesec, blindtext, color}
\definecolor{gray75}{gray}{0.75}
\newcommand{\hsp}{\hspace{20pt}}
\titleformat{\chapter}[hang]{\Huge\bfseries}{\thechapter\hsp\textcolor{gray75}{|}\hsp}{0pt}{\Huge\bfseries}
\renewcommand{\familydefault}{\sfdefault}
\usepackage[math]{iwona}

\addbibresource{../../library/bibliography.bib}

\title{EPO-2: Mid-term Design Report - Director}
\author{}

\begin{document}

\chapter{Director}
\label{ch:director}
Het programma wat voor de besturing van de robot zorgt op de computer, door ons "Director" gedoopt, bestaat zoals eerst bescheven uit vier hoofdonderdelen. De laatste drie worden hier beschreven.
\begin{itemize}
\item De pathfinder op basis van het A* algoritme
\item De event-driven seriële implentatie
\item De controller
\item De gebruikersinterface en visualizatie
\end{itemiize}
\section{Implementatie}
\label{sec:dirImplementatie}
We hebben uiteindelijk, zoals te lezen is in hoofdstuk \ref{ch:route}, gekozen voor een routeplanner gebaseerd op A*. Deze compilen we naar een standaard C dll (stdcall).

Ons hoofdproject wordt geschreven in C\# (.NET 4.5) en in combinatie met WPF (een UI framework gebaseerd op XAML UI's).
Dit geeft een vriendelijkere ontwikkelomgeving, managed talen zijn nu eenmaal wat vriendelijker voor de programmeur.
Het is ook mogelijk om alleen een van de dll's te vervangen, zodat iemand daar los aan kan werken.
Het hebben van een UI oogt prettiger en het gebruik van het eventmodel is heel fijn voor de seriële communicatie.
Dit heeft deels met persoonlijke voorkeur te maken, maar om de code voor iedereen in de projectgroep duidelijk te houden, hebben we de navigatie in pure C/C++ code gehouden.
De communicatie is eveneens geïmplementeerd in C\#, dit gaf het voordeel van een event-driven model.

\subsection{Communicatie}
TODO: Data.com module latex
\subsection{Controller}
TODO: Data.ctr module latex
\subsection{Gebruikersinterface en visualizatie}
TODO: Data.vis module latex
\end{document}