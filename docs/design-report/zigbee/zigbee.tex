\documentclass{report}
% Include all project wide packages here.
\usepackage{fullpage}
\usepackage[style=ieee]{biblatex}
\usepackage[dutch]{babel}
\usepackage[T1]{fontenc}
\usepackage{titlesec, blindtext, color}
\definecolor{gray75}{gray}{0.75}
\newcommand{\hsp}{\hspace{20pt}}
\titleformat{\chapter}[hang]{\Huge\bfseries}{\thechapter\hsp\textcolor{gray75}{|}\hsp}{0pt}{\Huge\bfseries}
\renewcommand{\familydefault}{\sfdefault}
\usepackage[math]{iwona}

\addbibresource{../../library/bibliography.bib}

\title{EPO-2: Mid-term Design Report - Draadloze Communicatie met ZigBee Module}
\author{Tijmen Witte}

\begin{document}

\chapter{Draadloze Communicatie met ZigBee Module}
\label{ch:probleem}
Een van de opdrachten van het project is om de robot te begeleiden via een C-programma. Het was de bedoeling dat we door middel van het draadloze communicatie systeem van ZigBee de robot lieten communiceren met de computer. Dit werd gedaan door één Xbee-module aan te sluiten aan de computer en deze te laten communiceren met de andere Xbee-module die al gemonteerd zat op het FPGA-bord van de robot.
\newline
\newline
X-bee modules zijn in feite niks anders dan digitale wifi systemen die op een 2.4 Ghz bandbreedte werken. Ze hebben een bereik van circa 30 tot 60 meter, waardoor ze makkelijk met andere apparaten kunnen communiceren die ook op een 2.4Ghz bandbreedte zitten. Daarnaast wordt de X-bee module aangestuurd door een 3.3 Voltage,verder kan het aantal signaalwisselingen per seconde nog worden ingesteld, hier wordt later nog verder op ingegaan.





\section{Eisen}

\section{Ontwerp}

\section{implementatie}

\section{test}

\section{discussie}



\end{document}