\documentclass{report}
% Include all project wide packages here.
\usepackage{fullpage}
\usepackage[style=ieee]{biblatex}
\usepackage[dutch]{babel}
\usepackage[T1]{fontenc}
\usepackage{titlesec, blindtext, color}
\definecolor{gray75}{gray}{0.75}
\newcommand{\hsp}{\hspace{20pt}}
\titleformat{\chapter}[hang]{\Huge\bfseries}{\thechapter\hsp\textcolor{gray75}{|}\hsp}{0pt}{\Huge\bfseries}
\renewcommand{\familydefault}{\sfdefault}
\usepackage[math]{iwona}

\addbibresource{../../library/bibliography.bib}

\title{EPO-2: Mid-term Design Report - Bijlagen}
\author{Robin Hes}

\begin{document}

\chapter{Bijlagen}
\label{ch:bijlagen}

\section{Wedstrijdveld}
\label{sec:field}

\begin{figure}[H]
	\centering
	\includegraphics[width=0.8\textwidth]{competitionField2440x2440-rc.pdf}
	\caption{Het wedstrijdveld te gebruiken bij EPO-2}
	\label{fig:field}
\end{figure}

\newpage
\section{Plan van aanpak}
\label{sec:pva}

\includepdf[pages={1-10}]{plan-van-aanpak-rc.pdf}

 \section{Routeplanner Pseudocode}
 \label{sec:pseudocode}

\subsection{Lee}
\label{ssec:pseudocode-lee}
% \begin{verbatim}
%  - Select start point, mark with 0
%  - i := 0
%   - REPEAT
%      - Mark all unlabeled neighbors of points marked with i with i+1
%      - i := i+1
%    UNTIL ((target reached) or (no points can be marked))
%    - go to the target point
%    REPEAT
%      - go to next node that has a lower mark than the actual node
%      - add this node to path
%    UNTIL (start point reached)
%    - Block the path for future wirings
%  - Delete all marks
% \end{verbatim}

\subsection{A*}
\label{ssec:pseudocode-astar}

\end{document}