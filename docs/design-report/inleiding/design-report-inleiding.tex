\documentclass{report}
% Include all project wide packages here.
\usepackage{fullpage}
\usepackage[style=ieee]{biblatex}
\usepackage[dutch]{babel}

\renewcommand{\familydefault}{\sfdefault}

\setmainfont[Ligatures=TeX]{Myriad Pro}
\setmathfont{Asana Math}
\setmonofont{Lucida Console}

\usepackage{titlesec, blindtext, color}
\definecolor{gray75}{gray}{0.75}
\newcommand{\hsp}{\hspace{20pt}}
\titleformat{\chapter}[hang]{\Huge\bfseries}{\thechapter\hsp\textcolor{gray75}{|}\hsp}{0pt}{\Huge\bfseries}
\renewcommand{\familydefault}{\sfdefault}
\renewcommand{\arraystretch}{1.2}
\setlength\parindent{0pt}

%For code listings
\definecolor{black}{rgb}{0,0,0}
\definecolor{browntags}{rgb}{0.65,0.1,0.1}
\definecolor{bluestrings}{rgb}{0,0,1}
\definecolor{graycomments}{rgb}{0.4,0.4,0.4}
\definecolor{redkeywords}{rgb}{1,0,0}
\definecolor{bluekeywords}{rgb}{0.13,0.13,0.8}
\definecolor{greencomments}{rgb}{0,0.5,0}
\definecolor{redstrings}{rgb}{0.9,0,0}
\definecolor{purpleidentifiers}{rgb}{0.01,0,0.01}


\lstdefinestyle{csharp}{
language=[Sharp]C,
showspaces=false,
showtabs=false,
breaklines=true,
showstringspaces=false,
breakatwhitespace=true,
escapeinside={(*@}{@*)},
columns=fullflexible,
commentstyle=\color{greencomments},
keywordstyle=\color{bluekeywords}\bfseries,
stringstyle=\color{redstrings},
identifierstyle=\color{purpleidentifiers},
basicstyle=\ttfamily\small}

\lstdefinestyle{c}{
language=C,
showspaces=false,
showtabs=false,
breaklines=true,
showstringspaces=false,
breakatwhitespace=true,
escapeinside={(*@}{@*)},
columns=fullflexible,
commentstyle=\color{greencomments},
keywordstyle=\color{bluekeywords}\bfseries,
stringstyle=\color{bluestrings},
identifierstyle=\color{purpleidentifiers}
}

\lstdefinestyle{vhdl}{
language=VHDL,
showspaces=false,
showtabs=false,
breaklines=true,
showstringspaces=false,
breakatwhitespace=true,
escapeinside={(*@}{@*)},
columns=fullflexible,
commentstyle=\color{greencomments},
keywordstyle=\color{bluekeywords}\bfseries,
stringstyle=\color{redstrings},
identifierstyle=\color{purpleidentifiers}
}

\lstdefinestyle{xaml}{
language=XML,
showspaces=false,
showtabs=false,
breaklines=true,
showstringspaces=false,
breakatwhitespace=true,
escapeinside={(*@}{@*)},
columns=fullflexible,
commentstyle=\color{greencomments},
keywordstyle=\color{redkeywords},
stringstyle=\color{bluestrings},
tagstyle=\color{browntags},
morestring=[b]",
  morecomment=[s]{<?}{?>},
  morekeywords={xmlns,version,typex:AsyncRecords,x:Arguments,x:Boolean,x:Byte,x:Char,x:Class,x:ClassAttributes,x:ClassModifier,x:Code,x:ConnectionId,x:Decimal,x:Double,x:FactoryMethod,x:FieldModifier,x:Int16,x:Int32,x:Int64,x:Key,x:Members,x:Name,x:Object,x:Property,x:Shared,x:Single,x:String,x:Subclass,x:SynchronousMode,x:TimeSpan,x:TypeArguments,x:Uid,x:Uri,x:XData,Grid.Column,Grid.ColumnSpan,Click,ClipToBounds,Content,DropDownOpened,FontSize,Foreground,Header,Height,HorizontalAlignment,HorizontalContentAlignment,IsCancel,IsDefault,IsEnabled,IsSelected,Margin,MinHeight,MinWidth,Padding,SnapsToDevicePixels,Target,TextWrapping,Title,VerticalAlignment,VerticalContentAlignment,Width,WindowStartupLocation,Binding,Mode,OneWay,xmlns:x}
}

%defaults
\lstset{
basicstyle=\ttfamily\small,
extendedchars=false,
numbers=left,
numberstyle=\ttfamily\tiny,
stepnumber=1,
tabsize=4,
numbersep=5pt
}
\addbibresource{../../library/bibliography.bib}

\title{EPO-2: Mid-term Design Report - Inleiding}
\author{Robin Hes}

\begin{document}

\chapter{Inleiding}
\label{ch:inleiding}

Nog steeds komen elke dag mensen om het leven door mijnen. Of het nu gaat om oorlogsgebieden of oud-oorlogsgebieden, het gevaar van een landmijn wijkt niet van zelf. Voor het project in het tweede semester van het eerste jaar in de studie Electrical Engineering, EE1810 of EPO-2, zijn we dan ook aan de slag gegaan met de automatische identificatie van mijnen. We hebben gedurende het tweede semester gewerkt aan een robot die in staat is autonoom een, door de computer berekende, route over een uit zwarte lijnen bestaand veld te rijden en tegelijkertijd in staat is mijnen (metalen schijfjes) te detecteren en hier op te reageren. Ons ontwerp en die van de andere projectgroepen, zullen getest worden in een wedstrijd waarin steeds uitdagendere challenges moeten worden uitgevoerd.
\\

\noindent
In deze verslaglegging zetten wij, groep D2, uiteen welke problemen wij tegen zijn gekomen, hoe we deze opgelost hebben, welke verdere keuzes we hebben gemaakt en hoe dit alles samenkomt in het uiteindelijke ontwerp.

Na deze inleiding volgt de probleemstelling, waarin het door de projectorganisatie opgelegde programma van eisen wordt uitgelicht. Dan volgt het systeemoverzicht, waarin het gehele ontwerp getoond wordt samen met een beknopte beschrijving.
Vervolgens is komt de robot uitgebreid aan bod. We gaan achtereenvolgens in op mechaniek die het mogelijk maakt dat de robot een zwarte lijn kan volgen; de eigenschappen van het wedstrijdveld en hoe wij deze gebruikt hebben voor positiebepaling en het volgen van de route; de manier waarop wij mijnen detecteren en de resulterende schakeling; de logica die het mogelijk maakt de servomotoren in de robot aan te sturen; de module die draadloze communicatie mogelijk maakt; een display waarop informatie wordt weergegeven die het makkelijker maakt fouten op te sporen; en, tot slot, het overkoepelende systeem dat al het voorgaande aanstuurt en controleert.

Vervolgens gaan we in op het brein van het systeem: de PC. Ten eerste beschrijven we de module die de door de robot te rijden route berekent, vervolgens de  grafische interface die interactiviteit met de gebruiker mogelijk maakt en de communicatie met de robot afhandelt, de \textit{Director}.

We sluiten dan af met een discussie, reflectie en conclusie.


\end{document}