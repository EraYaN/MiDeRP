\documentclass{report}
% Include all project wide packages here.
\usepackage{fullpage}
\usepackage[style=ieee]{biblatex}
\usepackage[dutch]{babel}
\usepackage[T1]{fontenc}
\usepackage{titlesec, blindtext, color}
\definecolor{gray75}{gray}{0.75}
\newcommand{\hsp}{\hspace{20pt}}
\titleformat{\chapter}[hang]{\Huge\bfseries}{\thechapter\hsp\textcolor{gray75}{|}\hsp}{0pt}{\Huge\bfseries}
\renewcommand{\familydefault}{\sfdefault}
\usepackage[math]{iwona}

\addbibresource{../../library/bibliography.bib}

\title{EPO-2: Mid-term Design Report - Zeven segmentendisplay}
\author{Joris Blom}

\begin{document}

\chapter{7-segmentendisplay}
\label{ch:sseg}
\section{Eisen}
\label{sec:Eisen}
Het 7-segmentdisplay moet aan de volgende eisen voldoen: 
\begin{itemize}
\item Het 7-segmentendisplay moet doormiddel van een 16 bits input de hexadecimale cijfers 0 t/m F op het elk 7-segmenten-display kunnen weergeven.
\item Het 7-segmentendislplay moet doormiddel van een 4 bits input de punten van elk 7-segmentendisplay los kunnen aansturen.
\end{itemize}
\section{Ontwerp}
De controller stuurt een 16 bits vector en een 4 bits vector naar de displayaansturing. De 16 bits vector stuurt de 7 segmenten aan en de 4 bits vector de punt onder het 7-segmentendisplay. De eerste 4 bits van de 16 bits vector.

Alle leds per 7-segmentendisplay hebben een gezamenlijke voedingsspanning, dus alle 8 anodes per display zijn elektrisch verbonden. Hierdoor kan elk 7-segmentendisplay afzonderlijk aan en uit worden geschakeld. De kathodes van de 
\begin{figure}[H]
\centering
\includegraphics{7_segment_display_schematic.png}

\caption{7-segmentendislplay}
\end{figure}


\section{Implementatie}
\section{Test}
\section{Discussie}

\end{document}