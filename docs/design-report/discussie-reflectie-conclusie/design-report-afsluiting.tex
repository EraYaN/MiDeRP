\documentclass{report}
% Include all project wide packages here.
\usepackage{fullpage}
\usepackage[style=ieee]{biblatex}
\usepackage[dutch]{babel}
\usepackage[T1]{fontenc}
\usepackage{titlesec, blindtext, color}
\definecolor{gray75}{gray}{0.75}
\newcommand{\hsp}{\hspace{20pt}}
\titleformat{\chapter}[hang]{\Huge\bfseries}{\thechapter\hsp\textcolor{gray75}{|}\hsp}{0pt}{\Huge\bfseries}
\renewcommand{\familydefault}{\sfdefault}
\usepackage[math]{iwona}

\addbibresource{../../library/bibliography.bib}

\begin{document}

\chapter{Conclusie}
\label{ch:conclusie}
%Werk de eisen uit (beantwoord eisen), see 3.2.3 (probleemstelling)
%-De robot moet een lijn kunnen volgen.
%-De robot moet een route kunnen volgen.
%-De robot moet een mijn kunnen detecteren,   onthouden waar deze ligt en vervolgens de mijn ontwijken.
%-De robot moet kunnen communiceren met de computer en andersom.

%BACKGROUND INFO/PROBLEEM
Het doel van het EPO-2 project is het ontwerpen van een autonoom mijndetecterende robot. In een vooraf bepaald veld moet de robot de mijnen in kaart brengen en ook ontwijken. 
%OPLOSSING (ALGORITME)
Om dit te realiseren moet een algoritme worden gebruikt om de kortste pad te bepalen en de mijnen in kaart te brengen. We hebben van het A*-algoritme gebruikt gemaakt die door middel van een heuristische schatting de kortste weg bepaald. 

%SUPPORT VOOR OPLOSSING
Dit algoritme was niet onze eerste keuze. Het Lee algoritme was eerst uitgewerkt, omdat het eenvoudig te begrijpen en implementeren is. Echter geeft Lee niet altijd de snelste route, want het algoritme houdt geen rekening met het aantal bochten dat het doorloopt.\\

%SYSTEEM (regel 19 tm 22: eis - mijnen opslaan)
De routeplanner in C is maar een klein deel van het systeem. De twee hoofdblokken van het systeem zijn: het programma dat op de computer draait en het programma op de robot.  

Op de computer zit de director wat voor de besturing van de robot zorgt. De robot stuurt een byte naar de computer, wat bijvoorbeeld het vragen om de volgende beweging kan zijn. In dat geval kan er door middel van de navigatie, het A*-algoritme, een route worden bepaald en vervolgens de volgende beweging naar de robot worden gestuurd. Wanneer de robot een mijn detecteert, wordt deze aan de lijst van mijnen toegevoegd en naar het unmanaged geheugen gekopieerd. Daarmee wordt er een nieuw pad berekend voor de robot.

Op de robot zit de main FSM die voor de detectie en beweging zorgen. Door middel van de drie reflectiesensoren kan de robot een lijn volgen. In de FSM gaat de robot steeds naar een andere state, afhankelijk van de navigatie in de director. De robot vraagt de computer om een commando en wacht totdat hij weer een commando terug krijgt voordat hij gaat rijden. Deze commando's zijn voor de beweging van de robot. Als er een mijn wordt gedetecteerd door middel van de inductieve sensor, dan zal de robot achteruit rijden totdat hij bij een kruispunt komt en wacht dan weer op het volgende commando.

Om de computer en robot met elkaar te laten communiceren, wordt er gebruikt van XBee modules die werken door middel van UART. Op de computer en robot zitten deze modules en hiermee wordt een byte van het een naar het ander verzonden.
\\

%MIJNDETECTOR (eis - mijn detecteren)
Om de mijnen op het wedstrijdveld te detecteren, wordt een inductieve sensor gebruikt. De inductieve sensor werkt op basis van de verandering van permeabiliteit. Wanneer de spoel 2.15 cm \pm 0.05 cm van de mijn is (center to center), dan wordt er een mijn gedetecteerd.\\

%PLAN VAN AANPAK
Om alle taken te kunnen volbrengen is het noodzakelijk om taken in te delen. Hierbij hebben we een plan van aanpak gemaakt, waar de taakverdelingen en deadlines instaan. Bovendien cre\"eert het een gestructureerde manier van denken voor elk projectlid.
%REFLECTIE, DISCUSSIE
Hoewel alles gepland kan worden, gaat toch niet alles zoals gedacht. Miscommunicatie met een projectlid heeft er toe geleid dat de gemaakte afspraken niet volledig waren doorgegeven. En het feit dat simulaties nooit de realiteit weergeven, gaf een tweetal veel problemen.

Het werken in tweetallen heeft voordelen en nadelen. Het zorgt ervoor dat problemen op verschillende manieren worden opgelost. Doordat er diverse oplossingen zijn, kan er een oplossing worden gekozen die het beste past. Echter leidt dit er wel toe dat als een subgroep vastzit, de andere subgroepen hun moeilijk kan helpen.  

\end{document}