\documentclass{report}
% Include all project wide packages here.
\usepackage{fullpage}
\usepackage[style=ieee]{biblatex}
\usepackage[dutch]{babel}
\usepackage[T1]{fontenc}
\usepackage{titlesec, blindtext, color}
\definecolor{gray75}{gray}{0.75}
\newcommand{\hsp}{\hspace{20pt}}
\titleformat{\chapter}[hang]{\Huge\bfseries}{\thechapter\hsp\textcolor{gray75}{|}\hsp}{0pt}{\Huge\bfseries}
\renewcommand{\familydefault}{\sfdefault}
\usepackage[math]{iwona}

\addbibresource{../../library/bibliography.bib}

\title{EPO-2: Mid-term Design Report - Discussie}
\author{Luc Does}

\begin{document}

\chapter{Discussie}
\label{ch:discussie}

In het EPO-1 project was er voor de groepen weinig keuze over het ontwerp van het audiosysteem, dit verandert in het EPO-2 project waarin op diverse manieren het eindresultaat bereikt kan worden. Hierdoor is het al te belangrijker de opties goed te bekijken en de uiteindelijke beslissing weloverwogen te nemen op basis van alle betrokken argumenten. \newline

Tot nu toe is er slechts een duidelijke keuze gemaakt over een onderdeel van de robot, en dat is de toegepaste routeplanner. De keuze is gevallen op het A* algoritme van Robin. Deze keuze is gebaseerd op de modulariteit en efficiëntie van de code. Ten opzichte van de andere mogelijke implementaties was deze variant makkelijker aan te passen in de toekomst en deze was ook het meest compleet. \newline

Een subtielere plek waar constant keuzes gemaakt moeten worden zijn de VHDL-beschrijvingen van de robot. Aangezien de functionaliteit van de robot op meerdere manieren beschreven kan worden moeten wij voor ons zelf bedenken wat de meest efficiënte, meest modulaire of simpelste oplossing zou zijn. Aangezien we dit doen in subgroepen zal er uiteindelijk een keuze gemaakt moeten worden tussen deze verschillende varianten. \newline

\end{document}