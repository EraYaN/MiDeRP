\documentclass{report}
% Include all project wide packages here.
\usepackage{fullpage}
\usepackage[style=ieee]{biblatex}
\usepackage[dutch]{babel}
\usepackage[T1]{fontenc}
\usepackage{titlesec, blindtext, color}
\definecolor{gray75}{gray}{0.75}
\newcommand{\hsp}{\hspace{20pt}}
\titleformat{\chapter}[hang]{\Huge\bfseries}{\thechapter\hsp\textcolor{gray75}{|}\hsp}{0pt}{\Huge\bfseries}
\renewcommand{\familydefault}{\sfdefault}
\usepackage[math]{iwona}

\addbibresource{../../library/bibliography.bib}

\title{EPO-2: Mid-term Design Report - Discussie}
\author{Luc Does}

\begin{document}

\chapter{Discussie}
\label{ch:discussie}

In het EPO-1 project was er voor de groepen weinig keuze over het ontwerp van het audiosysteem. Dit verandert in het EPO-2 project waarin op diverse manieren het eindresultaat bereikt kan worden. Hierdoor is het des te belangrijker de opties goed te bekijken en de uiteindelijke beslissing weloverwogen te nemen op basis van alle betrokken argumenten. \newline

Er zijn meerdere keuzes gemaakt met betrekking tot onderdelen van de robot. De meest belangrijke keuzes zijn het gebruikte type sensor, het algoritme van de routeplanner en de VHDL-beschrijving van de robot. \newline

Bij de routeplanner in C is de keuze  gevallen op het A*-algoritme van Robin. Deze keuze is gebaseerd op de modulariteit en efficiëntie van de code. Ten opzichte van de andere mogelijke implementaties was deze variant makkelijker aan te passen in de toekomst en deze was ook het meest compleet. \newline

De VHDL van Erwin heeft een plaats op de FPGA gekregen vanwege een uitgebreid systeem van states waarin de snelheid van de individuele servo's gemakkelijk aangepast kan worden. Door deze eigenschappen was dit een goede keuze voor de robot aangezien in de latere stadia vele verschillende states met verschillende eigenschappen geïmplementeerd moeten worden. \newline

De keuze tussen de verschillende type sensoren was tamelijk beperkt. Onze kennis reikt alleen tot simpele inductieve, resistieve en capacitieve sensoren. Onze eerste keuze was gevallen op de capacitieve sensor. In de JITs bleek deze gevoeliger als metaaldetector dan de inductieve variant. Helaas bleek het lastig om de schakeling werkend te krijgen met de beschikbare onderdelen op de robot. Hierna verschoof de keuze naar de inductieve sensor. Deze heeft het voordeel dat hij gemakkelijk te bouwen is en uit slechts enkele onderdelen bestaat; waar de capacitieve sensor wel drie opamps nodig had heeft de inductiesensor slechts een enkele opamp nodig. \newline

Een andere plek, waar de keuzes vaak veel subtieler zijn en er een stuk meer gemaakt moeten worden, is de VHDL-beschrijving van de robot. Aangezien de functionaliteit van de robot op meerdere manieren beschreven kan worden, moeten wij voor onszelf bedenken wat de meest efficiënte, meest modulaire of simpelste oplossing zou zijn. Vooral op het punt van stateflow moeten er veel keuzes gemaakt worden. Daarnaast zullen we bij het kalibreren van de robot moeten kiezen voor aanpassingen die simpel en klein zijn, maar een enkel probleem (gedeeltelijk) oplossen of wellicht een verandering in de code die ingrijpend is maar wel een groot deel van de problemen kan verhelpen. Bij ingrijpende aanpassingen loop je altijd het risico dat je meer kapot maakt dat verhelp.

\end{document}