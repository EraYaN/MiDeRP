\documentclass{article}
% Include all project wide packages here.
\usepackage{fullpage}
\usepackage[style=ieee]{biblatex}
\usepackage[dutch]{babel}

\renewcommand{\familydefault}{\sfdefault}

\setmainfont[Ligatures=TeX]{Myriad Pro}
\setmathfont{Asana Math}
\setmonofont{Lucida Console}

\usepackage{titlesec, blindtext, color}
\definecolor{gray75}{gray}{0.75}
\newcommand{\hsp}{\hspace{20pt}}
\titleformat{\chapter}[hang]{\Huge\bfseries}{\thechapter\hsp\textcolor{gray75}{|}\hsp}{0pt}{\Huge\bfseries}
\renewcommand{\familydefault}{\sfdefault}
\renewcommand{\arraystretch}{1.2}
\setlength\parindent{0pt}

%For code listings
\definecolor{black}{rgb}{0,0,0}
\definecolor{browntags}{rgb}{0.65,0.1,0.1}
\definecolor{bluestrings}{rgb}{0,0,1}
\definecolor{graycomments}{rgb}{0.4,0.4,0.4}
\definecolor{redkeywords}{rgb}{1,0,0}
\definecolor{bluekeywords}{rgb}{0.13,0.13,0.8}
\definecolor{greencomments}{rgb}{0,0.5,0}
\definecolor{redstrings}{rgb}{0.9,0,0}
\definecolor{purpleidentifiers}{rgb}{0.01,0,0.01}


\lstdefinestyle{csharp}{
language=[Sharp]C,
showspaces=false,
showtabs=false,
breaklines=true,
showstringspaces=false,
breakatwhitespace=true,
escapeinside={(*@}{@*)},
columns=fullflexible,
commentstyle=\color{greencomments},
keywordstyle=\color{bluekeywords}\bfseries,
stringstyle=\color{redstrings},
identifierstyle=\color{purpleidentifiers},
basicstyle=\ttfamily\small}

\lstdefinestyle{c}{
language=C,
showspaces=false,
showtabs=false,
breaklines=true,
showstringspaces=false,
breakatwhitespace=true,
escapeinside={(*@}{@*)},
columns=fullflexible,
commentstyle=\color{greencomments},
keywordstyle=\color{bluekeywords}\bfseries,
stringstyle=\color{bluestrings},
identifierstyle=\color{purpleidentifiers}
}

\lstdefinestyle{vhdl}{
language=VHDL,
showspaces=false,
showtabs=false,
breaklines=true,
showstringspaces=false,
breakatwhitespace=true,
escapeinside={(*@}{@*)},
columns=fullflexible,
commentstyle=\color{greencomments},
keywordstyle=\color{bluekeywords}\bfseries,
stringstyle=\color{redstrings},
identifierstyle=\color{purpleidentifiers}
}

\lstdefinestyle{xaml}{
language=XML,
showspaces=false,
showtabs=false,
breaklines=true,
showstringspaces=false,
breakatwhitespace=true,
escapeinside={(*@}{@*)},
columns=fullflexible,
commentstyle=\color{greencomments},
keywordstyle=\color{redkeywords},
stringstyle=\color{bluestrings},
tagstyle=\color{browntags},
morestring=[b]",
  morecomment=[s]{<?}{?>},
  morekeywords={xmlns,version,typex:AsyncRecords,x:Arguments,x:Boolean,x:Byte,x:Char,x:Class,x:ClassAttributes,x:ClassModifier,x:Code,x:ConnectionId,x:Decimal,x:Double,x:FactoryMethod,x:FieldModifier,x:Int16,x:Int32,x:Int64,x:Key,x:Members,x:Name,x:Object,x:Property,x:Shared,x:Single,x:String,x:Subclass,x:SynchronousMode,x:TimeSpan,x:TypeArguments,x:Uid,x:Uri,x:XData,Grid.Column,Grid.ColumnSpan,Click,ClipToBounds,Content,DropDownOpened,FontSize,Foreground,Header,Height,HorizontalAlignment,HorizontalContentAlignment,IsCancel,IsDefault,IsEnabled,IsSelected,Margin,MinHeight,MinWidth,Padding,SnapsToDevicePixels,Target,TextWrapping,Title,VerticalAlignment,VerticalContentAlignment,Width,WindowStartupLocation,Binding,Mode,OneWay,xmlns:x}
}

%defaults
\lstset{
basicstyle=\ttfamily\small,
extendedchars=false,
numbers=left,
numberstyle=\ttfamily\tiny,
stepnumber=1,
tabsize=4,
numbersep=5pt
}

\title{EPO-2: Persoonlijke evaluatie - Robin Hes}
\author{Robin Hes}

\begin{document}
\maketitle

\section*{Inleiding}

(Zelf)Evaluatie is helaas een belangrijk onderdeel van ieder leerproces en project en dus moeten we er bij EPO-2 helaas ook aan geloven. In dit artikel ga ik daarom in op het derde kwartaal en zal ik bespreken wat ik geleerd heb en wat ik denk bijgedragen te hebben aan ons gezamenlijke resultaat. Daarna komt nog een stukje evaluatie over het project in zijn algemeen: waren alle activiteiten nuttig? Wat was goed en wat kon beter? Als laatste volgt een korte conclusie.

\section*{Persoonlijk}

Ik heb een hoop geleerd tijdens het derde kwartaal. Vooral op het gebied van VHDL moest er een heleboel gebeuren en aangezien dat grotendeels nieuw was, viel er ook een hoop te leren. Ik zou het project dan ook wel willen omschrijven als een hele nuttige extensie van het vak Digitale Systemen. Hiernaast ben ik natuurlijk ook veel bezig geweest met de programmeertaal C, om niet alleen een implementatie van Lee's algoritme te maken, maar ook een van het A*-algoritme. Een kort maar zinvol intermezzo tussen al dit digitale werk was de JIT over op-amps. Het kostte me even om het principe van de relaxatie-oscillator te begrijpen, maar na het schrijven van het gevraagde verslag heb ik toch het idee dat ik aardig op de hoogte ben.

Ook op het gebied van groepswerk viel er (helaas) nog wat te leren: sommige mensen hebben behoorlijk wat aansporing nodig om te doen wat er van ze gevraagd wordt en zelfs dan hoeft er nog niet noodzakelijk iets te gebeuren. Het werk kwam in dit geval op de schouders van de rest van de groep en dat is wat mij betreft jammer. Iets waar we in projectverband helaas mee moeten leren leven.

Dan mijn eigen bijdrage. Ik heb het routeplanner-algoritme wat we gaan gebruiken in het eindresultaat geschreven, het een en ander in VHDL gebouwd en mijn steentje bijgedragen aan het merendeel van de ingeleverde verslagen. Ik vind dat ik mijn best heb gedaan om het groepsresultaat naar een aardig niveau te helpen, maar ik heb me daar niet voor uit de naad hoeven te werken, dit was naar mijn idee ook niet nodig, in dit kwartaal. Daarnaast hoop ik het team zo nu en dan wat sturing gegeven te hebben waar dat nodig was. Niet iedereen heeft hier iets mee gedaan, maar over het algemeen ben ik best tevreden.

\section*{Algemeen}

Op de vraag of ik alle activiteiten zinvol vond zou ik ``ja'' antwoorden, met grote uitzondering van de training informatievaardigheden. Ik denk dat er van uit gegaan mag worden dat iedereen die de studie Elektrotechniek komt volgen capabel genoeg is om een zoekopdracht in te tikken op Google en exacte instructies over welk type bron je inm welke database kunt vinden, blijven toch niet hangen. Vooral niet bij het gebrek aan interesse wat de meest studenten, waaronder ikzelf, zullen hebben als ze aan de opdracht beginnen.
Daarnaast vond ik het wat jammer dat alle VHDL-opdrachten per tweetal uitgevoerd moesten worden. Het argument ``iedereen moet het uiteindelijk toch kunnen'' is dan wel valide, maar niet iedere opdracht vereiste een unieke skillset en kostte toch een hoop tijd. Ik denk dat het beter zou zijn als er op toegezien zou worden dat iedereen een evenredige bijdrage levert aan slechts één VHDL-beschrijving. Dit zou een aanzienlijke hoeveelheid tijd besparen die vervolgens gebruikt kan worden voor verdere diepgang in en tijdens het project. Als laatste zou ik willen aanraden om voor volgend jaar de projecthandleiding, het liefst in één stuk, al vanaf middag 1 beschikbaar te hebben. Dit zou wat verwarring besparen bij de opstart. Ik heb er vertrouwen in dat dit goed komt.

\section*{Conclusie}

Kort samengevat: ik heb een hoop geleerd, zowel technisch als op het gebied van werken in project- en groepsverband. Niet alles ging vanzelf maar dat hoort er allemaal bij.
Verder is het project wat mij betreft klaar voor enkele kleine verbeteringen, maar de algemene indruk is tot nu toe goed.

\end{document}