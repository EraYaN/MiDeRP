\documentclass{article}
% Include all project wide packages here.
\usepackage{fullpage}
\usepackage[style=ieee]{biblatex}
\usepackage[dutch]{babel}
\usepackage[T1]{fontenc}
\usepackage{titlesec, blindtext, color}
\definecolor{gray75}{gray}{0.75}
\newcommand{\hsp}{\hspace{20pt}}
\titleformat{\chapter}[hang]{\Huge\bfseries}{\thechapter\hsp\textcolor{gray75}{|}\hsp}{0pt}{\Huge\bfseries}
\renewcommand{\familydefault}{\sfdefault}
\usepackage[math]{iwona}


\title{EPO-2: Persoonlijke evaluatie - Erwin de Haan}
\author{Erwin de Haan}

\begin{document}
\maketitle
Ik zal hier eens even een super mooie evaluatie maken, het is per slot van rekening mijn hobby. Daar gaan we.
\section*{Evaluatie}
In dit project zijn we met veel nieuwe technieken geconfronteerd, zoals Xilinx, een FPGA en VHDL bijvoorbeeld. Ik heb vooral veel extra kennis over VHDL opgedaan en bepaalde concepten van C zijn toch nog weer een keertje lekker herhaald. Ook heb ik geleerd hoe het mogelijk is om verschillende programmeertechnieken te combineren.\\

\noindent
De just-in-time trainingen hadden het zelfde effect als men gewoon gezegd had hier heb je het ontwerp voor je relaxatie oscillator, schrijf er een verslag over. Het was zeker leuk om even te klieren met al die draadjes, maar je was langer bezig om de schakeling goed te krijgen dan met het door proberen begrijpen van de schakeling.\\

\noindent
Ik heb uiteindelijk het halve "meten aan de robot"\:verslag gemaakt. Ook heb ik de GUI gemaakt voor ons uiteindelijke director programma, maar dat vind ik ook gewoon erg leuk. Het grote deel van onze (Robin \& Ik) VHDL code heb ik geschreven. Voor de rest heb ik de losse onderdelen van het systeem samengesmeed tot één werkend geheel. En natuurlijk het systeem overzicht in de mid-term report maar dat is alleen als ons (Robin \& Ik) systeem het systeem wordt, maar ik kreeg de impressie dat het het systeem was wat het verst was in de ontwikkeling.\\

\noindent
Het gebruik van git voor een project dat met meer dan twee man gerunt is vooral in het tweede deel van het kwartaal belangrijk geworden, men moet dus echt de functies van git gaan gbruiken om te verkomen dat er rare dingen gebeuren. Git heeft zo nu en dan de neiging om zichzelf om zeep te helpen. Dit is niet voor iedereen even makkelijk om op te vangen. Hetzefde geldt voor \LaTeX\:sommige vinden het nog steeds erg moeilijk om stukken in \LaTeX\:aan te leveren. Deze knikjes in de verslaglegging moeten echt glad gestreken worden voor het laatste kwartaal. We vergaderen nu eens in de week, het was toch fijner om gewoon iedere middag te vergaderen. Dan kun je toch veel beter overleggen tussen projectleden, dan zit men dus veel meer op één lijn.\\

\noindent
De coördinatie kan wel beter in onze projectgroep, daarin zouden die vergadingen natuurlijk ook helpen. Ook kan de "synchronisatie"\:tussen de tweetallen beter zodat er niet meer van die grote gaten tussen vallen.

\section*{Andere opmerkingen}
Ik heb het gevoel dat sommige doel efficienter berijkt kunnen worden. Wij heben nu bijvoorbeeld een aantal keer exact het zelfde werk gedaan. Dit zou natuurlijk makkelijker kunnen. Al zal het wel weer ervoor zorgen dat mensen meeliften. Volgend jaar zal het veel makkelijker zijn voor de studenten, omdat ze de module helemaal en bij elkaar hebben. Dat geeft een veel beter overzicht en dan vergeet je ook minder makkelijk dingen te lezen, zoals een opdrachtbeschrijving ;).



\end{document}