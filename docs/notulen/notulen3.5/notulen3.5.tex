\documentclass{article}
% Include all project wide packages here.
\usepackage{fullpage}
\usepackage[style=ieee]{biblatex}
\usepackage[dutch]{babel}

\begin{document}

\title{Notulen EPO-2 D-2 week 3.5}%<---WEEK NR
\author{Chy Lau}%<----AUTEUR
\maketitle

\section*{Informatie}
Datum van vergadering: 12 maart 2013\\ %<----DATUM
Locatie van vergadering: TU Delft - Drebbelweg zaal 1.060 %<----LOCATIE
\subsection*{Genodigden}
\begin{center}
\begin{tabular}{|c |c |c |c |}
	%%VUL AANWEZIGHEID IN
	\hline
	Joris Blom & Aanwezig & Tijmen Witte & Aanwezig \\
	\hline
	Luc Does & Aanwezig & Sander de Graaf (tutor) & Aanwezig \\
	\hline
	Erwin de Haan & Aanwezig & Pascal 't Hart (studentassistent) & Aanwezig \\
	\hline
	Robin Hes (voorzitter) & Aanwezig & Vincent Voogt (studentassistent) & Aanwezig \\
	\hline
	Chy Lau (notulist) & Aanwezig & & \\
	\hline
\end{tabular}
\end{center}

\section*{Vergadering}
\begin{enumerate}
	
%%%%%%%%%%%%%%%%%%%%%%%
%----- VOORAF
\subsection*{Vooraf}
%OPENINGSTIJD
\item Vergadering geopend om 13:35

%%NOTULEN VORIGE VERGADERING
\item Notulen vorige vergadering
\begin{itemize}
\item Alle punten in de notulen moeten duidelijk worden opgeschreven. Als men later de notulen weer doorleest, zal ieder groepslid een helder beeld hebben van wat er is besproken.
\end{itemize}

\item De agenda wordt ongewijzigd vastgesteld.
%%MEDEDELINGEN
\item Mededelingen
\begin{itemize}
\item Op maandag 18 maart moet heel D-2 (en ook alle andere groepen) naar de TN-Studio classroom 2 - A 053. De just-in-time training Opamps zal daar plaatsvinden en het zal beginnen om 13:30 tot 17:30.
\end{itemize}

%%%%%%%%%%%%%%%%%%%%%%%
%----- DEADLINES
\subsection*{Deadlines}
%%KOMENDE DEADLINES
\item Komende deadlines
\begin{itemize}
\item De 2e versie van het meetrapport 'Robot als afstandmeter' moet op donderdag 14 maart worden ingeleverd bij de tutor.
\item In week 3.6 moet er een meetrapport van de JIT: Opamps worden ingeleverd bij de tutor. 
\end{itemize}

%%%%%%%%%%%%%%%%%%%%%%%
%----- ACTIES
\subsection*{Actiepunten}
%%TODO VANDAAG
\item Todo vandaag
\begin{itemize}
\item De groepsleden die nog niet klaar zijn met de robot schakeling (opdrachten van middag 1 en 2, week 3.4) moeten deze eerst afmaken. Daarna moet er aan de seriële en draadloze communicatie met behulp van UART en ZigBee gewerkt worden. 
\end{itemize}

%%TODO LATER
\item Todo later
\begin{itemize}
\item Voor de volgende keer moet de handleiding JIT: UART / XBee worden doorgelezen.
\item Volgende week donderdag (week 3.6) kan er nog gewerkt worden aan de robot schakeling en de seriële en draadloze communicatie met behulp van UART en ZigBee. 
\end{itemize}

%%%%%%%%%%%%%%%%%%%%%%%
%----- BESLUITEN
\subsection*{Besluiten}
\item Genomen besluiten
\begin{itemize}
%%GENOMEN BESLUITEN
\item Meetrapport 'Robot als afstandmeter': Luc helpt Joris met het verbeteren van het meetrapport.
\end{itemize}

%%%%%%%%%%%%%%%%%%%%%%%
%----- AFSLUITING
\noindent 
\subsection*{Afsluitend}
%<----SLUITINGSTIJD
\item Vergadering gesloten om 13:52

\end{enumerate}

\end{document}