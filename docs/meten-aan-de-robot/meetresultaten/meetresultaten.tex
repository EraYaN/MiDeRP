\documentclass{report}
% Include all project wide packages here.
\usepackage{fullpage}
\usepackage[style=ieee]{biblatex}
\usepackage[dutch]{babel}
\usepackage[T1]{fontenc}
\usepackage{titlesec, blindtext, color}
\definecolor{gray75}{gray}{0.75}
\newcommand{\hsp}{\hspace{20pt}}
\titleformat{\chapter}[hang]{\Huge\bfseries}{\thechapter\hsp\textcolor{gray75}{|}\hsp}{0pt}{\Huge\bfseries}
\renewcommand{\familydefault}{\sfdefault}
\usepackage[math]{iwona}


\begin{document}
\chapter{Resultaten en Discussie}
Er zijn drie verschillende metingen verricht, kalibratiemeting, onbekende afstandsmeting en de lineariteitsmeting. Deze metingen zijn drie keer gedaan en dus zijn de meetresultaten ook drie keer weergeven en geanalyseerd.
\section{Kalibratie}
Hieronder zijn de meetresultaten van de eerste kalibratiemeting te zien. Bij deze meting is het de bedoeling dat deze 10 keer wordt uitgevoerd, maar bij 1 groepje is de opdracht verkeerd ge\"interpreteerd en is de meting maar vijf keer uitgevoerd.\\
\small{De gemiddelde tijd die de robot over de kalibratie afstand doet.}
\begin{equation}
\label{eq:avgTime}
t_{gem}=\frac{\sum_{i=1}^{n}t_i}{n}
\end{equation}
\small{De gemiddelde rijsnelheid van de robot.}
\begin{equation}
\label{eq:avgVel}
v_{gem} = \frac{s}{t_{gem}}
\end{equation}
\small{De standaarddeviatie van de tijd.}
\begin{equation}
\label{eq:standaardDev}
\sigma(t) = \sqrt{\frac{\sum_{i=1}^{n}( t_i-t_{gem})^2}{n-1}}
\end{equation}
\small{De absolute fout van de tijd.}
\begin{equation}
\label{eq:timeError}
u(t) = \frac{s}{\sqrt{n}} = \sqrt{\frac{\sum_{i=1}^{n}( t_i-t_{gem})^2}{n(n-1)}}
\end{equation}
\small{De absolute fout van de snelheid.}
\begin{equation}
\label{eq:velError}
u(v) = \sqrt{\left (\frac{\partial v }{\partial t }\right)^2 u(t)^2 + \left (\frac{\partial v }{\partial s }\right)^2 u(s)^2} = \sqrt{\left (\frac{s }{{t_{gem}}^2 }\right)^2 u(t)^2 + \left (\frac{1}{t_{gem} }\right)^2 u(s)^2}
\end{equation}

Als we de functies \ref{eq:avgTime},  \ref{eq:avgVel},  \ref{eq:standaardDev},  \ref{eq:timeError} en  \ref{eq:velError} op de meetresultaten (Tabel \ref{tab:measurementsCalib}) loslaten krijgen we de waarden als in Tabel \ref{tab:resCali}.

\begin{figure}[H]
\includegraphics[width=0.33\textwidth,page=2]{grafiekmeetresultaten1}
\includegraphics[width=0.33\textwidth,page=1]{grafiekmeetresultaten2}
\includegraphics[width=0.33\textwidth,page=3]{grafiekmeetresultaten3}
\caption{De meetresultaten in grafieken, hieruit is duidelijk de speiding van de verschillende metingen te zien. \\ \small{ Van links naar rechts Robin en Erwin, Joris en Chy, Luc en Tijmen}}
\label{fig:measureGraph}
\end{figure}

\begin{table}
\centering
\caption{Meetresulataten kalibratie}
\label{tab:measurementsCalib}
\subfloat[Robin en Erwin]{\begin{tabular}{| l| c|}
\hline
   Meting \# & t (s) \\
\hline
   1& 1.031 \\
\hline
   2& 1.042 \\
\hline
   3& 1.042 \\
\hline
   4& 1.046 \\
\hline
   5& 1.045 \\
\hline
\end{tabular}
}
\quad
\subfloat[Joris en Chy]{\begin{tabular}{| l| c|}
\hline
   Meting \# & t (s) \\
\hline
   1& 1.128\\
\hline
   2& 1.127\\
\hline
   3& 1.137\\
\hline
  4& 1.142\\
\hline
   5& 1.130 \\
\hline
   6& 1.136\\
\hline
   7& 1.140\\
\hline
   8& 1.130\\
\hline
   9& 1.131\\
\hline
   10& 1.145 \\
\hline
\end{tabular}
}
\quad
\subfloat[Luc en Tijmen]{\begin{tabular}{| l| c|}
\hline
   Meting \# & t(s) \\
\hline
   1& 1.170 \\
\hline
   2& 1.173 \\
\hline
   3& 1.142 \\
\hline
   4& 1.146 \\
\hline
   5& 1.164 \\
\hline
   6& 1.167 \\
\hline
   7& 1.171 \\
\hline
   8& 1.167\\
\hline
   9& 1.156\\
\hline
   10& 1.152\\
\hline
\end{tabular}
}
\end{table}

\begin{table}
\centering
\caption{Eigenschappen Robot}
\label{tab:resCali}
\subfloat[Robin en Erwin]{\begin{tabular}{| l| c|}
\hline
   Naam & Waarde \\
\hline
   $t_{gem}$ & 1.041 s \\
\hline
   $v_{gem}$ & 9.602 cm/s \\
\hline
   $\sigma(t)$ & 0.582 s \\
\hline
   $u(t_{gem})$ & 0.003 s  \\
\hline
   $u(v_{gem})$ & 0.054 cm/s \\
\hline
\end{tabular}
}
\quad
\subfloat[Joris en Chy]{\begin{tabular}{| l| c|}
\hline
   Naam & Waarde \\
\hline
   $t_{gem}$ & 1.135 s \\
\hline
   $v_{gem}$ & 8.813 cm/s \\
\hline
   $\sigma(t)$ & 0.006 s \\
\hline
   $u(t_{gem})$ & 0.002 s  \\
\hline
   $u(v_{gem})$ & 0.047 cm/s \\
\hline
\end{tabular}
}
\quad
\subfloat[Luc en Tijmen]{\begin{tabular}{| l| c|}
\hline
   Naam & Waarde \\
\hline
   $t_{gem}$ & 1.161 s \\
\hline
   $v_{gem}$ & 8.614 cm/s \\
\hline
   $\sigma(t)$ & 0.011 s \\
\hline
   $u(t_{gem})$ & 0.004 s  \\
\hline
   $u(v_{gem})$ & 0.050 cm/s \\
\hline
\end{tabular}
}
\end{table}

\section{Onbekende afstand}

In Tabel \ref{tab:resExtra}a zijn de meetresultaten te zien van de onbekende afstandsmeting. Om de snelheid te kunnen bepalen moet er een afstand bekend zijn. Deze afstand is gemeten en deze is 23.4 cm.

\begin{table}
\centering
\caption{De resultaten van de onbekende afstandsmetingen en de lineairiteitsmetingen.}
\label{tab:resExtra}
\subfloat[Onbekende afstand]{\begin{tabular}{| l| c|}
\hline
  Meting \#  & t (s)\\
\hline
  R\&E & 2.526 \\
\hline
J\&C & 2.664 \\
\hline
L\&T & 2.741 \\
\hline
 \end{tabular}
}
\quad
\subfloat[Lineairiteit]{\begin{tabular}{| l| c| c| c| c| c|}
\hline
  Meting \# & $t_{5cm} (s)$ & $t_{10cm} (s)$ & $t_{15cm} (s)$ & $t_{20cm} (s)$ & $t_{25cm} (s)$\\
\hline
  R\&E & 0.542 & 1.092 & 1.638 & 2.218 & 2.728 \\
\hline
J\&C & 0.575 & 1.140 & 1.732 & 2.330 & 2.854 \\
\hline
L\&T & 0.584 & 1.177 & 1.789 & 2.375 & 2.955 \\
\hline
 \end{tabular}
}
\end{table}

\section{Lineariteit}
\subsection*{Meting 1}
In Tabel \ref{tab:resExtra}b zijn de meetresultaten te zien van de lineariteitsmeting. Doormiddel van de meetresultaten en de afstand is de gemiddelde snelheid uitgerekend. De grafieken in Figuur \ref{fig:measureGraph} geven een duidelijk beeld van de metingen.

\begin{figure}[H]
\includegraphics[width=0.33\textwidth,page=2]{Lineariteitsmeting1}
\includegraphics[width=0.33\textwidth,page=1]{Lineariteitsmeting2}
\includegraphics[width=0.33\textwidth,page=3]{Lineariteitsmeting3}
\caption{De lineairitijdsmetingen en de uitgerekende snelheid van de kalibratie in 1 grafiek. \\ \small{ Van links naar rechts Robin en Erwin, Joris en Chy, Luc en Tijmen}}
\label{fig:measureGraph}
\end{figure}

\newpage
\chapter{Conclusie}
Het meetresultaat van dit meetrapport is goed te gebruiken in de rest van ons project. Dit is mede te danken aan de hoge kwaliteit van de meetresultaten. De onzekerheid en de afwijkingen zijn erg klein bij alle metingen die zijn verricht. De gemiddelde snelheid van de robot is ongeveer 9,6 cm/s. Bij de lineariteitsmeting was de gemiddelde snelheid vrijwel onafhankelijk van de afstand, en dus is de afstandsmeter van de robot erg lineair. Tijdens het rijden maakt de robot een bocht met een radius van ongeveer 80 cm. Hiermee moet rekening gehouden worden bij het programmeren van de robot. Het feit dat de afstandsmeter van de robot erg lineair is, is een gunstige conclusie voor ons project.
\end{document}



\end{document}