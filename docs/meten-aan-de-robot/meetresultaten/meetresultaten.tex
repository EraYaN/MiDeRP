\documentclass{report}
% Include all project wide packages here.
\usepackage{fullpage}
\usepackage[style=ieee]{biblatex}
\usepackage[dutch]{babel}
\usepackage[T1]{fontenc}
\usepackage{titlesec, blindtext, color}
\definecolor{gray75}{gray}{0.75}
\newcommand{\hsp}{\hspace{20pt}}
\titleformat{\chapter}[hang]{\Huge\bfseries}{\thechapter\hsp\textcolor{gray75}{|}\hsp}{0pt}{\Huge\bfseries}
\renewcommand{\familydefault}{\sfdefault}
\usepackage[math]{iwona}


\title{Resultaten}
\author{J.F.J. Blom}
\date{\today}
\begin{document}
\chapter{Resultaten en Discussie}
Er zijn vier verschillende metingen verricht, kalibratiemeting, onbekende afstandsmeting en de lineariteitsmeting. In dit hoofstuk zijn alle meetresultaten weergeven en geanalyseerd.Van alle meetresultaten is het gemiddelde genomen, doormiddel van deze formule:
\begin{equation}
\bar{x}=\frac{\sum_{i=1}^{n}x_i}{n}
\end{equation}

Hier is $x_i$ een meetresultaat en n is het aantal metingen. 
Daarnaast is de onzekerheid van de metingen berekend. Daarbij is gebruik gemaakt van de volgende formule:
\begin{equation}
u=\sqrt{\frac{\sum_{i=1}^{n}( x_i-x_a)^2}{n(n-1)}}
\end{equation}
$x_a$ is hier het gemiddelde van de meting.
\subsection*{Kalibratie}
Hieronder zijn de meetresultaten van de kalibratiemeting te zien. Drie groepen hebben metingen verricht waarvan de groep van Robin en Erwin dezelfde robot heeft gebruikt als de groep van Tijmen en Luc.

\begin{table}
 \centering
\begin{tabular}{| l| c|}
\hline
    & Kalibratie 10 cm \\
\hline
   Meting 1 (s) & 1.031 \\
\hline
   Meting 2 (s) & 1.042 \\
\hline
   Meting 3 (s) & 1.042 \\
\hline
   Meting 4 (s) & 1.046 \\
\hline
   Meting 5 (s) & 1.045 \\
\hline
   Gemiddelde (s) & 1.041 \\
\hline
   Gemiddelde snelheid (cm/s) & 9.602 \\
\hline
   Onzekerheid tijd (s) & 0.003 \\
\hline
   Onzekerheid snelheid (cm/s) & 0.025 \\
\hline
\end{tabular}
\caption{Tabel meetresultaten kalibratie Joris en Chy}
\end{table}

$$ t_{gem}=\frac{\sum_{i=1}^{n}t_i}{n} = 1.041 s$$

$$ v_{gem} = \frac{s}{t_{gem}} = 9.602 cm/s$$

$$ u(t) = \frac{s}{\sqrt{n}} = \sqrt{\frac{\sum_{i=1}^{n}( t_i-t_{gem})^2}{n(n-1)}} = 0.003 s$$ 

%berekening voor onzekerheid moet nog geplaatst worden.

\begin{table}
 \centering
\begin{tabular}{| l| c|}
\hline
    & Kalibratie 10 cm \\
\hline
   Meting 1 (s) & 1,164 \\
\hline
   Meting 2 (s) & 1.170 \\
\hline
   Meting 3 (s) & 1.173 \\
\hline
   Meting 4 (s) & 1.142 \\
\hline
   Meting 5 (s) & 1.146 \\
\hline
   Meting 6 (s) & 1.167 \\
\hline
   Meting 7 (s) & 1.171 \\
\hline
   Meting 8 (s) & 1.167 \\
\hline
   Meting 9 (s) & 1.156 \\
\hline
   Meting 10 (s) & 1.152 \\
\hline
   Gemiddelde (s) & 1.161 \\
\hline
   Gemiddelde snelheid (cm/s) & 8.614 \\
\hline
   Onzekerheid tijd (s) & 1.208E-5
 \\
\hline
   Onzekerheid snelheid (cm/s) & 1.040E-4 \\
\hline
\end{tabular}
\caption{Tabel meetresultaten kalibratie Luc en Tijmen}
\end{table}

$$ t_{gem}=\frac{\sum_{i=1}^{n}t_i}{n} = 1.161 s$$

$$ v_{gem} = \frac{s}{t_{gem}} = 8.614 cm/s$$

$$ u = \frac{s}{\sqrt{n}} = \sqrt{\frac{\sum_{i=1}^{n}( t_i-t_{gem})^2}{n(n-1)}} = 1.208E-5 s$$

\begin{figure}[H]
 \centering
\includegraphics[width=150mm] {grafiekmeetresultaten.jpg}
\caption{Grafiek meetresultaten kalibratie Joris en Chy}
\end{figure}
Hierboven zijn de meetresultaten van de kalibratiemeting in een diagram gezet. Hier zijn goed de verschillen tussen de metingen te zien. De metingen liggen redelijk dicht bij elkaar, en dus is de kwaliteit van deze meting erg hoog.

\section{Lineariteit}
Hieronder zijn de meetresultaten te zien van de onbekende afstandsmeting. Het doel is om met behulp van de eerder berekende snelheid de afstand te meten. Om de afstand te berekenen wordt gebruik gemaakt van de formule

\begin{equation}
s = v \cdot t
\end{equation}

Ook hebben wij met behulp van een liniaal de werkelijke afstand bepaald. Deze bedroeg 23.4cm.
\newline

\begin{table}
 \centering
\begin{tabular}{| l| c|}
\hline
    & Onbekende afstand 23.4 cm\\
\hline
   Meting 1 (s) & 2.526 \\
\hline
   Meting 2 (s) & 2.531 \\
\hline
   Meting 3 (s) & 2.515 \\
\hline
   Meting 4 (s) & 2.518 \\
\hline
   Meting 5 (s) & 2.522 \\
\hline
   Gemiddelde (s) & 2.523 \\
\hline
   Gemiddelde snelheid (cm/s) & 9.276 \\
\hline
   Onzekerheid tijd (s) & 0.005 \\
\hline
   Onzekerheid snelheid (cm/s) & 0.007 \\
\hline
 \end{tabular}
\caption{Tabel meetresultaten onbekende afstand Joris en Chy}
\end{table}

Deze tijden kunnen we gebruiken in combinatie met de gemiddelde snelheid die we eerder hebben vastgesteld om de onbekende afstand te berekenen

$$ s = v_{gem} \cdot t_{gem}$$
$$ = 9.602 cm/s \cdot 2.523 s = 24.23 cm$$

\begin{table}
 \centering
\begin{tabular}{| l| c| c| c| c| c|}
\hline
   & 5 cm & 10 cm & 15 cm & 20 cm & 25 cm\\
\hline
   Meting 1 (s) & 0.542 & 1.092 & 1.638 & 2.218 & 2.728 \\
\hline
   Meting 2 (s) & 0.550 & 1.086 & 1.636 & 2.237 & 2.747 \\
\hline
   Meting 3 (s) & 0.547 & 1.089 & 1.627 & 2.210 & 2.728 \\
\hline
   Meting 4 (s) & 0.548 & 1.081 & 1.634 & 2.218 & 2.724 \\
\hline
   Meting 5 (s) & 0.542 & 1.081 & 1.658 & 2.233 & 2.733 \\
\hline
   Gemiddelde (s) & 0.546 & 1.086 & 1.638 & 2.223 & 2.732 \\
\hline
   Gemiddelde snelheid (cm/s) & 9.164 & 9.211 & 9.155 & 8.996 & 9.151 \\
\hline
   Onzekerheid tijd (s) & 0.002 & 0.002 & 0.005 & 0.005 & 0.004 \\
\hline
   Onzekerheid snelheid (cm/s) & 0.055 & 0.018 & 0.019 & 0.10 & 0.005 \\
\hline
 \end{tabular}
\caption{TODO caption}
\end{table}
\begin{figure}[H]
 \centering
\includegraphics[width=150mm] {afstand-tellerstand.jpg}
\caption{TODO Caption}
\end{figure}
In de bovenstaande grafiek is de afstand afgezet tegen de tellerstand van de robot. De lijn is bijna recht, dus de afstandsmeter van de robot is ook bijna lineair, en dus is de snelheid onafhankelijk van de afstand.  Op de volgende pagina zet ik de afstand af tegen de snelheid, waardoor de afwijkingen in lineariteit beter te zien zijn. 
\begin{figure}[H]
 \centering
\includegraphics[width=150mm] {afstand-snelheid.jpg}
\caption{TODO Caption}
\end{figure}
In de bovenstaande grafiek is de afstand afgezet tegen de snelheid. Hier zijn goed de afwijkingen in lineariteit van de afstandsmeter te zien. De snelheid blijft tussen de 9.22 en de 8.99, dus de snelheid is vrij constant. De afstandsmeter van de robot is dus behoorlijk lineair. De kleine afwijkingen kunnen ook liggen aan de vele andere factoren die de meting be"invloeden. De robot rijd bijvoorbeeld in een bocht met een radius van 80 cm. Daarnaast is de robot misschien niet altijd loodrecht ten opzichte van de startlijn gestart, waardoor de meting wordt be"invloed.
\newpage
\chapter{Conclusie}
Het meetresultaat van dit meetrapport is goed te gebruiken in de rest van ons project. Dit is mede te danken aan de hoge kwaliteit van de meetresultaten. De onzekerheid en de afwijkingen zijn erg klein bij alle metingen die zijn verricht. De gemiddelde snelheid van de robot is ongeveer 9,6 cm/s. Bij de lineariteitsmeting was de gemiddelde snelheid vrijwel onafhankelijk van de afstand, en dus is de afstandsmeter van de robot erg lineair. Tijdens het rijden maakt de robot een bocht met een radius van ongeveer 80 cm. Hiermee moet rekening gehouden worden bij het programmeren van de robot. Het feit dat de afstandsmeter van de robot erg lineair is, is een gunstige conclusie voor ons project.
\end{document}