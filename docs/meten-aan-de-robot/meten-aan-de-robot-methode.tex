\documentclass[a4paper]{article}
\usepackage{fullpage}
\usepackage{amsmath}
\usepackage{mathenv}
\usepackage[dutch]{babel} 
\usepackage{verbatim}

\begin{document}
\section{Methode}
\subsection{Benodigdheden}
De benodigdheden zijn als volgt:
\begin{itemize}
\item Robot met de programmatuur in de ROM van het FPGA bord
\item Papier met kalibratielijnen en lijnen op onbekende afstand van elkaar.
\item Liniaal
\end{itemize}
\subsection{Methode}
Eerst laat men de robot de kalibratie afstand vijf keer meten. Meten doet men door de robot voor de eerste lijn te zetten en dan op de start knop te drukken, het is het beste om hierbij de voorkant van de robot tegen de grond te duwen, de robot stopt vanzelf als hij de overkant heeft berijkt. Het aflezen gaat in 2 stappen eerst de eerste 4 cijfers en dan de laatste 4 cijfers, $XX.XXYYYY$, je kunt zien welke cijfers je ziet op de 4 zeven-segment-display's aan de hand van de aanwezigheid dan wel afwezigheid van de decimale punt. De robot die wij hebben gebruikt had een afwijking naar links. Hij maakte een cirkel met een straal tussen de 8 en 10 decimeter, als de afstand die je wilt meten dus vrij lang is moet je de robot misschien een beetje schuin voor de eerste lijn zetten.
Hierna kan men de onbekende afstand ook meten op de zelfde manier.
Nu moet de lineairiteit nog bepaald worden, dit kan door alle afstanden op het blad een aantal keer te meten.
\end{document}