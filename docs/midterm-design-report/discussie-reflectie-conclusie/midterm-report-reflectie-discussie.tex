\documentclass{report}
% Include all project wide packages here.
\usepackage{fullpage}
\usepackage[style=ieee]{biblatex}
\usepackage[dutch]{babel}

\renewcommand{\familydefault}{\sfdefault}

\setmainfont[Ligatures=TeX]{Myriad Pro}
\setmathfont{Asana Math}
\setmonofont{Lucida Console}

\usepackage{titlesec, blindtext, color}
\definecolor{gray75}{gray}{0.75}
\newcommand{\hsp}{\hspace{20pt}}
\titleformat{\chapter}[hang]{\Huge\bfseries}{\thechapter\hsp\textcolor{gray75}{|}\hsp}{0pt}{\Huge\bfseries}
\renewcommand{\familydefault}{\sfdefault}
\renewcommand{\arraystretch}{1.2}
\setlength\parindent{0pt}

%For code listings
\definecolor{black}{rgb}{0,0,0}
\definecolor{browntags}{rgb}{0.65,0.1,0.1}
\definecolor{bluestrings}{rgb}{0,0,1}
\definecolor{graycomments}{rgb}{0.4,0.4,0.4}
\definecolor{redkeywords}{rgb}{1,0,0}
\definecolor{bluekeywords}{rgb}{0.13,0.13,0.8}
\definecolor{greencomments}{rgb}{0,0.5,0}
\definecolor{redstrings}{rgb}{0.9,0,0}
\definecolor{purpleidentifiers}{rgb}{0.01,0,0.01}


\lstdefinestyle{csharp}{
language=[Sharp]C,
showspaces=false,
showtabs=false,
breaklines=true,
showstringspaces=false,
breakatwhitespace=true,
escapeinside={(*@}{@*)},
columns=fullflexible,
commentstyle=\color{greencomments},
keywordstyle=\color{bluekeywords}\bfseries,
stringstyle=\color{redstrings},
identifierstyle=\color{purpleidentifiers},
basicstyle=\ttfamily\small}

\lstdefinestyle{c}{
language=C,
showspaces=false,
showtabs=false,
breaklines=true,
showstringspaces=false,
breakatwhitespace=true,
escapeinside={(*@}{@*)},
columns=fullflexible,
commentstyle=\color{greencomments},
keywordstyle=\color{bluekeywords}\bfseries,
stringstyle=\color{bluestrings},
identifierstyle=\color{purpleidentifiers}
}

\lstdefinestyle{vhdl}{
language=VHDL,
showspaces=false,
showtabs=false,
breaklines=true,
showstringspaces=false,
breakatwhitespace=true,
escapeinside={(*@}{@*)},
columns=fullflexible,
commentstyle=\color{greencomments},
keywordstyle=\color{bluekeywords}\bfseries,
stringstyle=\color{redstrings},
identifierstyle=\color{purpleidentifiers}
}

\lstdefinestyle{xaml}{
language=XML,
showspaces=false,
showtabs=false,
breaklines=true,
showstringspaces=false,
breakatwhitespace=true,
escapeinside={(*@}{@*)},
columns=fullflexible,
commentstyle=\color{greencomments},
keywordstyle=\color{redkeywords},
stringstyle=\color{bluestrings},
tagstyle=\color{browntags},
morestring=[b]",
  morecomment=[s]{<?}{?>},
  morekeywords={xmlns,version,typex:AsyncRecords,x:Arguments,x:Boolean,x:Byte,x:Char,x:Class,x:ClassAttributes,x:ClassModifier,x:Code,x:ConnectionId,x:Decimal,x:Double,x:FactoryMethod,x:FieldModifier,x:Int16,x:Int32,x:Int64,x:Key,x:Members,x:Name,x:Object,x:Property,x:Shared,x:Single,x:String,x:Subclass,x:SynchronousMode,x:TimeSpan,x:TypeArguments,x:Uid,x:Uri,x:XData,Grid.Column,Grid.ColumnSpan,Click,ClipToBounds,Content,DropDownOpened,FontSize,Foreground,Header,Height,HorizontalAlignment,HorizontalContentAlignment,IsCancel,IsDefault,IsEnabled,IsSelected,Margin,MinHeight,MinWidth,Padding,SnapsToDevicePixels,Target,TextWrapping,Title,VerticalAlignment,VerticalContentAlignment,Width,WindowStartupLocation,Binding,Mode,OneWay,xmlns:x}
}

%defaults
\lstset{
basicstyle=\ttfamily\small,
extendedchars=false,
numbers=left,
numberstyle=\ttfamily\tiny,
stepnumber=1,
tabsize=4,
numbersep=5pt
}
\addbibresource{../../library/bibliography.bib}

\title{EPO-2: Mid-term Design Report - Reflectie en Discussie}
\author{Luc Does}

Een project uitvoeren verloopt vrijwel nooit vloeiend. Hoe ervaren de projectgroep is en hoe goed het project voorbereid is, er zal altijd wel iets misgaan. Ook ons project heeft de nodige tegenslagen te verduren gekregen, van deze tegenslagen kunnen we gelukkig wel veel leren. \newline

Bij het starten van een project moeten er altijd afspraken gemaakt worden. Wanneer dit niet gebeurt zijn er zaken onduidelijk en daarover kan de voortgang onder lijden. Ook bij ons project zijn er op de eerste projectdag afspraken gemaakt. Deze bevatten bijvoorbeeld het gebruik van \LaTeX, Github en de begin- en eindtijden van de projectmiddagen. Ook was er in de derde projectweek een standaardsjabloon voor documenten in \LaTeX. Het gebruik van \LaTeX en Github begon moeizaam maar iedereen deed zijn best om zo snel mogelijk de touwtjes in handen te nemen. \newline

Het is  vanzelfsprekend dat iedereen zijn best doet om zich aan de afspraken te houden, maar dan moeten die afspraken wel bekend zijn. Aangezien bijna alle afspraken op de eerste projectdag gemaakt en Joris die dag niet aanwezig was waren en nog enkele dingen onduidelijk. Dit leidde ertoe dat Joris van gedachte was dat de projectmiddagen om kwart voor twee begonnen. Dit had de groep duidelijk moeten maken, eveneens stond de informatie wel in de notulen. \newline

Ook de opzet van het project heeft voor ons voordelen en nadelen. Wij werken in groepen van twee aan een eigen implementatie van de lijnvolger. Dit zorgt ervoor dat wij problemen op diverse manieren oplossen en daardoor veel mogelijkheden voor de uiteindelijke robot hebben. Een voorbeeld hiervan is de C-code voor de routeplanner. Wij hebben vier verschillende implementaties gemaakt, waarvan het A* algoritme uiteindelijk is verkozen voor in de uiteindelijke lijnvolger. Maar de diversiteit zorgt er ook voor dat de implementaties zover verschillen dat wanneer er op een probleem gestuit wordt en een subgroep vastzit, een andere subgroep niet snel een oplossing kan bedenken. \newline

Een ander punt dat tijdens het project duidelijk is geworden is het verschil tussen simulatie en realiteit. Het wordt altijd al gezegd dat een simulatie geen garantie is voor de werking van het echte systeem. In EPO-1 benaderden de resultaten van de simulaties de werkelijkheid behoorlijk, maar tussen de simulatie en synthese van VHDL-code blijkt het verschil groter. \newline

Voor Tijmen en Luc gaf de simulatie van de werking van het gehele lijnvolgersysteem een correcte werking van het systeem weer, waarbij op basis van sensoren het PWM-signaal gegenereert wordt. Maar bij de implementatie van het systeem bleek slechts een enkele motor te draaien, waarbij de andere motoraansturing periodieke ruis verstuurde. 