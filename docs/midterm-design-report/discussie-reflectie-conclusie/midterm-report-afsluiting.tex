\documentclass{report}
% Include all project wide packages here.
\usepackage{fullpage}
\usepackage[style=ieee]{biblatex}
\usepackage[dutch]{babel}
\usepackage[T1]{fontenc}
\usepackage{titlesec, blindtext, color}
\definecolor{gray75}{gray}{0.75}
\newcommand{\hsp}{\hspace{20pt}}
\titleformat{\chapter}[hang]{\Huge\bfseries}{\thechapter\hsp\textcolor{gray75}{|}\hsp}{0pt}{\Huge\bfseries}
\renewcommand{\familydefault}{\sfdefault}
\usepackage[math]{iwona}

\addbibresource{../../library/bibliography.bib}

\begin{document}

\chapter{Conclusie}
\label{ch:conclusie}

%BACKGROUND INFO/PROBLEEM
Het doel van het EPO-2 project is het ontwerpen van een autonoom mijndetecterende robot. In een vooraf bepaald veld moet de robot de mijnen in kaart brengen en ook ontwijken. 
%OPLOSSING (ALGORITME)
Om dit te realiseren moet een algoritme worden gebruikt om de kortste pad te bepalen en de mijnen in kaart te brengen. We hebben van het A*-algoritme gebruikt gemaakt die door middel van een heuristische schatting de kortste weg bepaald. 

%SUPPORT VOOR OPLOSSING
Dit algoritme was niet onze eerste keuze. Het Lee algoritme was eerst uitgewerkt, omdat het eenvoudig te begrijpen en implementeren is. Echter geeft Lee niet altijd de snelste route, want het algoritme houdt geen rekening met het aantal bochten dat het doorloopt.\\

%SYSTEEMOVERZICHT

%PLAN VAN AANPAK
Om alle taken te kunnen volbrengen is het noodzakelijk om taken in te delen. Hierbij hebben we een plan van aanpak gemaakt, waarin de taakverdelingen en deadlines instaan. Bovendien cre\"eert het een gestructureerde manier van denken voor elk projectlid.
%REFLECTIE

%DISCUSSIE
	

\end{document}