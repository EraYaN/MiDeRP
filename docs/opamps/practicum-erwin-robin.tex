\documentclass{report}
% Include all project wide packages here.
\usepackage{fullpage}
\usepackage[style=ieee]{biblatex}
\usepackage[dutch]{babel}
\usepackage[T1]{fontenc}
\usepackage{titlesec, blindtext, color}
\definecolor{gray75}{gray}{0.75}
\newcommand{\hsp}{\hspace{20pt}}
\titleformat{\chapter}[hang]{\Huge\bfseries}{\thechapter\hsp\textcolor{gray75}{|}\hsp}{0pt}{\Huge\bfseries}
\renewcommand{\familydefault}{\sfdefault}
\usepackage[math]{iwona}


\title{Meetrapport: Opamps}
\author{Robin P. Hes\\\&\\Erwin R. de Haan}
\begin{document}
\maketitle
\chapter{Practicum}
\section{inverterende versterker}
\begin{enumerate}
\item Done
\item $A=-\frac{R_a}{R_b}=-\frac{10000}{1000}=-10$
\item Check
\item $\phi = \pi$
\item $V_{in,pp} = 220.0 \text{mV} \wedge V_{uit,pp} = 2.021 \text{V}$
\item $A=-\frac{V_{uit,pp}}{V_{in,pp}}=-\frac{2.021}{0.2200}=-9.186$
\item $f_{-3dB}=137.80000 \text{kHz}$
\item 

\begin{tabular}{|c|c|c|c|c|}
\hline
$R_a$ & berekende A & gemeten A bij 100Hz & -3dB frequentie & GBW\\
\hline
%1\text{k\Omega}
& 
1 
&
 $A=-\frac{V_{uit,pp}}{V_{in,pp}}=-\frac{0.1920}{0.2240}=-0.8571$ 
& 
1.1200000MHz
&
$960000$
\\
\hline
%$10\text{k\Omega}$
&10 & $A=-\frac{V_{uit,pp}}{V_{in,pp}}=-\frac{2.021}{0.2200}=-9.186$ & 137.80000kHz&$1265\cdot10^3$\\
\hline
%$100\text{k\Omega}$
 & 100 & $A=-\frac{V_{uit,pp}}{V_{in,pp}}=-\frac{19.90}{0.2200}=-90.45$&14.600000kHz&$1321\cdot10^3$\\
\hline
\end{tabular}
\item 9
\item 10

\end{enumerate}
\section{Integrator}
\begin{enumerate}
\item 1
\item 2
\item Done.
\item Zaagtand, $helling=\frac{V_{in,pp}}{f^-1}$
\item $helling=\frac{V}{s}=\frac{0.600}{30\cdot10^{-6}}=20000$
\item Lagere amplitude.
\item Zonder generator is uitgangssignaal nul, zonder Rd extreme ruis.
\item 100Hz, ongeveer. (thuis)
\item Het uitgangssignaal gaat steeds meer lijken op het geinverteerde ingangssignaal.
\end{enumerate}
\section{Comparator}
\begin{enumerate}
\item Done.
\item 12.80 V en 12.80 V
\item 3
\item 76mV omhoog en omlaag
\item Uitgangssignaal verandert niet.
\item 6
\end{enumerate}
\section{Relaxatie}
\begin{enumerate}
\item 1
\item 2
\item 3
\item 1.070 kHz
\item 69 kHz
\item Meer licht geeft een hogere frequentie en vice versa.
\end{enumerate}
\end{document}