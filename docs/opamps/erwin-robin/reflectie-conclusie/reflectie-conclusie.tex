\documentclass{report}
% Include all project wide packages here.
\usepackage{fullpage}
\usepackage[style=ieee]{biblatex}
\usepackage[dutch]{babel}
\usepackage[T1]{fontenc}
\usepackage{titlesec, blindtext, color}
\definecolor{gray75}{gray}{0.75}
\newcommand{\hsp}{\hspace{20pt}}
\titleformat{\chapter}[hang]{\Huge\bfseries}{\thechapter\hsp\textcolor{gray75}{|}\hsp}{0pt}{\Huge\bfseries}
\renewcommand{\familydefault}{\sfdefault}
\usepackage[math]{iwona}

\addbibresource{../../../library/bibliography.bib}

\title{Meetrapport: Opamps - Reflectie \& Conclusie}
\author{Robin Hes\\\&\\Erwin R. de Haan}

\begin{document}
\chapter{Reflectie \& Conslusie}
\section{Reflectie}
Het gebruik van ee relaxatie oscillator is een goede optie. De nauwkeurigheid van de meting hangt af van de nauwkeurigheid van het kloksignaal en van de verschillende componenten. Deze zijn allemaal bekend en men kan dus een uitspraak doen over de totale nauwkeurigheid. Het kloksignaal moet ook een dermate hoge frequentie hebben dat deze vele malen hoger is dan de frequentie gegenereerd door de oscillator. Voor de toepassing waarvoor wij deze zullen gebruiken, een mijnen detector, gaat het meer om veranderingen in het signaal dan om de waarde.
\section{Conclusie}
We hebben kunnen zie (Formule \ref{eq:freqOscillator}) dat de frequentie afhankelijk is van de weerstand van de sensor en de waarde van de capaciteit in de integrator. Wij kunnen dus door de capaciteit te varieren het bereik van de frequentie tweaken. Dit kan handig zijn om verschillende ordes van grootte van weerstanden te gebruiken.
\end{document}