\documentclass{report}
% Include all project wide packages here.
\usepackage{fullpage}
\usepackage[style=ieee]{biblatex}
\usepackage[dutch]{babel}
\usepackage[T1]{fontenc}
\usepackage{titlesec, blindtext, color}
\definecolor{gray75}{gray}{0.75}
\newcommand{\hsp}{\hspace{20pt}}
\titleformat{\chapter}[hang]{\Huge\bfseries}{\thechapter\hsp\textcolor{gray75}{|}\hsp}{0pt}{\Huge\bfseries}
\renewcommand{\familydefault}{\sfdefault}
\usepackage[math]{iwona}


\title{Meetrapport: Opamps - Inleiding}
\author{Robin Hes\\\&\\Erwin R. de Haan}

\begin{document}
\chapter{Inleiding}

\section{Inleiding}
In dit meetrapport zetten we de theorie achter de zogenaamde relaxatie-oscillator uit een, een apparaat dat het mogelijk maakt om een analoge sensor te laten communiceren met een digitaal systeem, zoals een FPGA. We zullen een vergelijking maken tussen theoretisch gedrag en daadwerkelijk gemeten waarden en een korte beschrijving geven van de praktische inzetbaarheid in combinatie met een FPGA. Ten slotte zullen we onze bevindingen samenvatten in een conclusie.

\end{document}