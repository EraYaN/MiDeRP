\documentclass{report}
% Include all project wide packages here.
\usepackage{fullpage}
\usepackage[style=ieee]{biblatex}
\usepackage[dutch]{babel}

\renewcommand{\familydefault}{\sfdefault}

\setmainfont[Ligatures=TeX]{Myriad Pro}
\setmathfont{Asana Math}
\setmonofont{Lucida Console}

\usepackage{titlesec, blindtext, color}
\definecolor{gray75}{gray}{0.75}
\newcommand{\hsp}{\hspace{20pt}}
\titleformat{\chapter}[hang]{\Huge\bfseries}{\thechapter\hsp\textcolor{gray75}{|}\hsp}{0pt}{\Huge\bfseries}
\renewcommand{\familydefault}{\sfdefault}
\renewcommand{\arraystretch}{1.2}
\setlength\parindent{0pt}

%For code listings
\definecolor{black}{rgb}{0,0,0}
\definecolor{browntags}{rgb}{0.65,0.1,0.1}
\definecolor{bluestrings}{rgb}{0,0,1}
\definecolor{graycomments}{rgb}{0.4,0.4,0.4}
\definecolor{redkeywords}{rgb}{1,0,0}
\definecolor{bluekeywords}{rgb}{0.13,0.13,0.8}
\definecolor{greencomments}{rgb}{0,0.5,0}
\definecolor{redstrings}{rgb}{0.9,0,0}
\definecolor{purpleidentifiers}{rgb}{0.01,0,0.01}


\lstdefinestyle{csharp}{
language=[Sharp]C,
showspaces=false,
showtabs=false,
breaklines=true,
showstringspaces=false,
breakatwhitespace=true,
escapeinside={(*@}{@*)},
columns=fullflexible,
commentstyle=\color{greencomments},
keywordstyle=\color{bluekeywords}\bfseries,
stringstyle=\color{redstrings},
identifierstyle=\color{purpleidentifiers},
basicstyle=\ttfamily\small}

\lstdefinestyle{c}{
language=C,
showspaces=false,
showtabs=false,
breaklines=true,
showstringspaces=false,
breakatwhitespace=true,
escapeinside={(*@}{@*)},
columns=fullflexible,
commentstyle=\color{greencomments},
keywordstyle=\color{bluekeywords}\bfseries,
stringstyle=\color{bluestrings},
identifierstyle=\color{purpleidentifiers}
}

\lstdefinestyle{vhdl}{
language=VHDL,
showspaces=false,
showtabs=false,
breaklines=true,
showstringspaces=false,
breakatwhitespace=true,
escapeinside={(*@}{@*)},
columns=fullflexible,
commentstyle=\color{greencomments},
keywordstyle=\color{bluekeywords}\bfseries,
stringstyle=\color{redstrings},
identifierstyle=\color{purpleidentifiers}
}

\lstdefinestyle{xaml}{
language=XML,
showspaces=false,
showtabs=false,
breaklines=true,
showstringspaces=false,
breakatwhitespace=true,
escapeinside={(*@}{@*)},
columns=fullflexible,
commentstyle=\color{greencomments},
keywordstyle=\color{redkeywords},
stringstyle=\color{bluestrings},
tagstyle=\color{browntags},
morestring=[b]",
  morecomment=[s]{<?}{?>},
  morekeywords={xmlns,version,typex:AsyncRecords,x:Arguments,x:Boolean,x:Byte,x:Char,x:Class,x:ClassAttributes,x:ClassModifier,x:Code,x:ConnectionId,x:Decimal,x:Double,x:FactoryMethod,x:FieldModifier,x:Int16,x:Int32,x:Int64,x:Key,x:Members,x:Name,x:Object,x:Property,x:Shared,x:Single,x:String,x:Subclass,x:SynchronousMode,x:TimeSpan,x:TypeArguments,x:Uid,x:Uri,x:XData,Grid.Column,Grid.ColumnSpan,Click,ClipToBounds,Content,DropDownOpened,FontSize,Foreground,Header,Height,HorizontalAlignment,HorizontalContentAlignment,IsCancel,IsDefault,IsEnabled,IsSelected,Margin,MinHeight,MinWidth,Padding,SnapsToDevicePixels,Target,TextWrapping,Title,VerticalAlignment,VerticalContentAlignment,Width,WindowStartupLocation,Binding,Mode,OneWay,xmlns:x}
}

%defaults
\lstset{
basicstyle=\ttfamily\small,
extendedchars=false,
numbers=left,
numberstyle=\ttfamily\tiny,
stepnumber=1,
tabsize=4,
numbersep=5pt
}

\title{Meetrapport: Opamps}
\author{Robin P. Hes\\\&\\Erwin R. de Haan}
\begin{document}
\maketitle
\chapter{Practicum}
\section{inverterende versterker}
LM741
\begin{enumerate}
\item Done
\item $A=-\frac{R_a}{R_b}=-\frac{10000}{1000}=-10$
\item Check
\item $\phi = \pi$
\item $V_{in,pp} = 220.0 \mathrm{mV} \wedge V_{uit,pp} = 2.021 \mathrm{V}$
\item $A=-\frac{V_{uit,pp}}{V_{in,pp}}=-\frac{2.021}{0.2200}=-9.186$
\item $f_{-3dB}=137.80000 \mathrm{kHz}$
\item Tabel\\
\begin{tabular}{|c|c|c|c|c|}
\hline
$R_a$ & berekende A & gemeten A bij 100Hz & -3dB frequentie & GBW\\
\hline
$1 \mathrm{k}\Omega$& -1 & $A=-\frac{V_{uit,pp}}{V_{in,pp}}=-\frac{0.1920}{0.2240}=-0.8571$ & 1.1200000MHz&$(1-A)\cdot f_{-3dB}=2.080\cdot10^6 Hz$ \\
\hline
$10 \mathrm{k}\Omega$
&-10 & $A=-\frac{V_{uit,pp}}{V_{in,pp}}=-\frac{2.021}{0.2200}=-9.186$ & 137.80000kHz&$(1-A)\cdot f_{-3dB}=1.404\cdot10^6 Hz$\\
\hline
$100 \mathrm{k}\Omega$
 & -100 & $A=-\frac{V_{uit,pp}}{V_{in,pp}}=-\frac{19.90}{0.2200}=-90.45$&14.600000kHz&$(1-A)\cdot f_{-3dB}=1.335\cdot10^6 Hz$\\
\hline
\end{tabular}
\item Done. See 8.
\item De datasheet geeft 0.437 MHz als minimum 1.5 MHz als typisch. De op-amp op het bord voldoet dus aan deze eisen.

\end{enumerate}
\section{Integrator}
LF356
\begin{enumerate}
\item \begin{equation}
V_{out}(t) = V_{in}(t)\cdot \frac{\frac{1}{j\omega C}}{R_{pot}}
\end{equation}
\item \begin{equation}
A_v(s)=\frac{V_{out}(s)}{V_{in}(s)} = \frac{\frac{1}{sC}}{R_{pot}}
\end{equation}
\item Done.
\item Zaagtand, $helling=\frac{V_{in,pp}}{f^-1}$
\item $helling=\frac{V}{s}=\frac{0.600}{30\cdot10^{-6}}=20000$
\item Lagere amplitude.
\item Zonder generator is uitgangssignaal nul, zonder Rd extreme ruis.
\item 100Hz, ongeveer. (thuis)
\item Het uitgangssignaal gaat steeds meer lijken op het geinverteerde ingangssignaal.
\end{enumerate}
\section{Comparator}
LF356
\begin{enumerate}
\item Done.
\item 12.80 V en 12.80 V
\item \begin{equation}
V_{max}=V_{min} \to V_{drempel} = V_{max}\cdot\frac{R_d}{R_c+R_d}=20\cdot\frac{1000}{1000+10000}=1.81\mathrm{V}
\end{equation}
\item 76mV omhoog en omlaag
\item Uitgangssignaal verandert niet.
\item 6
\end{enumerate}
\section{Relaxatie}
\begin{enumerate}
\item 1
\item 2
\item 3
\item 1.070 kHz
\item 69 kHz
\item Meer licht geeft een hogere frequentie en vice versa.
\end{enumerate}
\end{document}