\documentclass{report}
% Include all project wide packages here.
\usepackage{fullpage}
\usepackage[style=ieee]{biblatex}
\usepackage[dutch]{babel}
\usepackage[T1]{fontenc}
\usepackage{titlesec, blindtext, color}
\definecolor{gray75}{gray}{0.75}
\newcommand{\hsp}{\hspace{20pt}}
\titleformat{\chapter}[hang]{\Huge\bfseries}{\thechapter\hsp\textcolor{gray75}{|}\hsp}{0pt}{\Huge\bfseries}
\renewcommand{\familydefault}{\sfdefault}
\usepackage[math]{iwona}

\addbibresource{../../../library/bibliography.bib}

\title{Meetrapport: Opamps - Resultaten}
\author{Robin Hes\\\&\\Erwin R. de Haan}

\begin{document}
\chapter{Resultaten}
\section{Metingen}
De gemeten frequentie met de pot-meter op de maximale waarde $f_{R_{pot,max}} = 1070 \mathrm{Hz}$. Als we de de componentwaarden (Tabel \ref{tab:components}) invullen in de formule voor de frequentie (Formule \ref{eq:freqOscillator}) krijgen we 1000 Hz. Het verschil tussen deze waarden (70 Hz) is een verschil te wijten aan de afwijkinginen in de potmeter en condensator. Deze waarden stemmen dus overeen.
\begin{table}[H]
\caption{De component waarden in de relaxatie oscillator.}
\label{tab:components}
\centering
\begin{tabular}{|c|c|}
\hline
Naam & Waarde\\
\hline
$R_a$ & $1 \mathrm{k\Omega}$\\
\hline
$R_b$ & $10 \mathrm{k\Omega}$\\
\hline
$R_c$ & $10 \mathrm{k\Omega}$\\
\hline
$R_d$ & $1 \mathrm{k\Omega}$\\
\hline
$R_{pot}$ & $0\--25 \mathrm{k\Omega}$\\
\hline
$C$ & $10 \mathrm{nF}$\\
\hline
\end{tabular}
\end{table}
\section{Mogelijke implementatie}
In de FPGA kunnen we het aantal klokpulsen tellen tussen twee events die het signaal hoog maken. We zouden dit zelfs kunnen middelen met het vorige resultaat.
\begin{equation}
\label{eq:fpgaEq}
R = \frac{N_{clk}}{4Cf_{FPGA}}
\end{equation}
Waarbij $f_{FPGA} = 50 \mathrm{MHz}$ en $N_{clk}$ is het aantal getelde klokcycli. Omdat de FPGA veel makkelijker rekent met integer waarden kunnen we de vergelijking nog maal $10^8$ doen om $C$ ook als integer te krijgen.
\end{document}