\documentclass{report}
% Include all project wide packages here.
\usepackage{fullpage}
\usepackage[style=ieee]{biblatex}
\usepackage[dutch]{babel}

\renewcommand{\familydefault}{\sfdefault}

\setmainfont[Ligatures=TeX]{Myriad Pro}
\setmathfont{Asana Math}
\setmonofont{Lucida Console}

\usepackage{titlesec, blindtext, color}
\definecolor{gray75}{gray}{0.75}
\newcommand{\hsp}{\hspace{20pt}}
\titleformat{\chapter}[hang]{\Huge\bfseries}{\thechapter\hsp\textcolor{gray75}{|}\hsp}{0pt}{\Huge\bfseries}
\renewcommand{\familydefault}{\sfdefault}
\renewcommand{\arraystretch}{1.2}
\setlength\parindent{0pt}

%For code listings
\definecolor{black}{rgb}{0,0,0}
\definecolor{browntags}{rgb}{0.65,0.1,0.1}
\definecolor{bluestrings}{rgb}{0,0,1}
\definecolor{graycomments}{rgb}{0.4,0.4,0.4}
\definecolor{redkeywords}{rgb}{1,0,0}
\definecolor{bluekeywords}{rgb}{0.13,0.13,0.8}
\definecolor{greencomments}{rgb}{0,0.5,0}
\definecolor{redstrings}{rgb}{0.9,0,0}
\definecolor{purpleidentifiers}{rgb}{0.01,0,0.01}


\lstdefinestyle{csharp}{
language=[Sharp]C,
showspaces=false,
showtabs=false,
breaklines=true,
showstringspaces=false,
breakatwhitespace=true,
escapeinside={(*@}{@*)},
columns=fullflexible,
commentstyle=\color{greencomments},
keywordstyle=\color{bluekeywords}\bfseries,
stringstyle=\color{redstrings},
identifierstyle=\color{purpleidentifiers},
basicstyle=\ttfamily\small}

\lstdefinestyle{c}{
language=C,
showspaces=false,
showtabs=false,
breaklines=true,
showstringspaces=false,
breakatwhitespace=true,
escapeinside={(*@}{@*)},
columns=fullflexible,
commentstyle=\color{greencomments},
keywordstyle=\color{bluekeywords}\bfseries,
stringstyle=\color{bluestrings},
identifierstyle=\color{purpleidentifiers}
}

\lstdefinestyle{vhdl}{
language=VHDL,
showspaces=false,
showtabs=false,
breaklines=true,
showstringspaces=false,
breakatwhitespace=true,
escapeinside={(*@}{@*)},
columns=fullflexible,
commentstyle=\color{greencomments},
keywordstyle=\color{bluekeywords}\bfseries,
stringstyle=\color{redstrings},
identifierstyle=\color{purpleidentifiers}
}

\lstdefinestyle{xaml}{
language=XML,
showspaces=false,
showtabs=false,
breaklines=true,
showstringspaces=false,
breakatwhitespace=true,
escapeinside={(*@}{@*)},
columns=fullflexible,
commentstyle=\color{greencomments},
keywordstyle=\color{redkeywords},
stringstyle=\color{bluestrings},
tagstyle=\color{browntags},
morestring=[b]",
  morecomment=[s]{<?}{?>},
  morekeywords={xmlns,version,typex:AsyncRecords,x:Arguments,x:Boolean,x:Byte,x:Char,x:Class,x:ClassAttributes,x:ClassModifier,x:Code,x:ConnectionId,x:Decimal,x:Double,x:FactoryMethod,x:FieldModifier,x:Int16,x:Int32,x:Int64,x:Key,x:Members,x:Name,x:Object,x:Property,x:Shared,x:Single,x:String,x:Subclass,x:SynchronousMode,x:TimeSpan,x:TypeArguments,x:Uid,x:Uri,x:XData,Grid.Column,Grid.ColumnSpan,Click,ClipToBounds,Content,DropDownOpened,FontSize,Foreground,Header,Height,HorizontalAlignment,HorizontalContentAlignment,IsCancel,IsDefault,IsEnabled,IsSelected,Margin,MinHeight,MinWidth,Padding,SnapsToDevicePixels,Target,TextWrapping,Title,VerticalAlignment,VerticalContentAlignment,Width,WindowStartupLocation,Binding,Mode,OneWay,xmlns:x}
}

%defaults
\lstset{
basicstyle=\ttfamily\small,
extendedchars=false,
numbers=left,
numberstyle=\ttfamily\tiny,
stepnumber=1,
tabsize=4,
numbersep=5pt
}
\addbibresource{../../../library/bibliography.bib}

\title{Meetrapport: Opamps - Resultaten}
\author{Robin Hes\\\&\\Erwin R. de Haan}

\begin{document}
\chapter{Resultaten}
\section{Metingen}
De gemeten frequentie met de pot-meter op de maximale waarde $f_{R_{pot,max}} = 1070 \mathrm{Hz}$. Als we de de componentwaarden (Tabel \ref{tab:components}) in vullen in de formule voor de frequentie (Formule \ref{eq:freqOscillator}) krijgen we 1000 Hz. Het verschil tussen deze waarden (70 Hz) is een verschil te wijten aan de afwijkinginen in de potmeter en condensator. Deze waarden stemmen dus overeen.
\begin{table}[H]
\caption{De gemeten waarden van de inverterende versterker.}
\label{tab:invAmpMeasured}
\centering
\begin{tabular}{|c|c|c|c|c|}
\hline
$R_a$ & berekende A & gemeten A bij 100Hz & -3dB frequentie & GBW\\
\hline
$1 \mathrm{k}\Omega$& -1 & $A=-\frac{V_{uit,pp}}{V_{in,pp}}=-\frac{0.1920}{0.2240}=-0.8571$ & 1.1200000MHz&$(1-A)\cdot f_{-3dB}=2.080\cdot10^6 Hz$ \\
\hline
$10 \mathrm{k}\Omega$
&-10 & $A=-\frac{V_{uit,pp}}{V_{in,pp}}=-\frac{2.021}{0.2200}=-9.186$ & 137.80000kHz&$(1-A)\cdot f_{-3dB}=1.404\cdot10^6 Hz$\\
\hline
$100 \mathrm{k}\Omega$
 & -100 & $A=-\frac{V_{uit,pp}}{V_{in,pp}}=-\frac{19.90}{0.2200}=-90.45$&14.600000kHz&$(1-A)\cdot f_{-3dB}=1.335\cdot10^6 Hz$\\
\hline
\end{tabular}
\end{table}
\begin{table}[H]
\caption{De component waarden in de relaxatie oscillator.}
\label{tab:components}
\centering
\begin{tabular}{|c|c|}
\hline
Naam & Waarde\\
\hline
$R_a$ & $1 \mathrm{k\Omega}$\\
\hline
$R_b$ & $10 \mathrm{k\Omega}$\\
\hline
$R_c$ & $10 \mathrm{k\Omega}$\\
\hline
$R_d$ & $1 \mathrm{k\Omega}$\\
\hline
$R_{pot}$ & $0\--25 \mathrm{k\Omega}$\\
\hline
$C$ & $10 \mathrm{nF}$\\
\hline
\end{tabular}
\end{table}
\section{Mogelijke implementatie}
In de FPGA kunnen we het aantal klokpulsen tellen tussen twee events die het signaal hoog maken. We zouden dit zelfs kunnen middelen met het vorige resultaat.
\begin{equation}
\label{eq:fpgaEq}
R = \frac{N_{clk}}{4Cf_{FPGA}}
\end{equation}
Waarbij $f_{FPGA} = 50 \mathrm{MHz}$ en $N_{clk}$ is het aantal getelde klokcycli. Omdat de FPGA veel makkelijker rekent met integer waarden kunnen we de vergelijking nog maal $10^8$ doen om $C$ ook als integer te krijgen.
\end{document}