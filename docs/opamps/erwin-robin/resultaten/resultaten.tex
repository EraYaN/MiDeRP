\documentclass{report}
% Include all project wide packages here.
\usepackage{fullpage}
\usepackage[style=ieee]{biblatex}
\usepackage[dutch]{babel}
\usepackage[T1]{fontenc}
\usepackage{titlesec, blindtext, color}
\definecolor{gray75}{gray}{0.75}
\newcommand{\hsp}{\hspace{20pt}}
\titleformat{\chapter}[hang]{\Huge\bfseries}{\thechapter\hsp\textcolor{gray75}{|}\hsp}{0pt}{\Huge\bfseries}
\renewcommand{\familydefault}{\sfdefault}
\usepackage[math]{iwona}

\addbibresource{../../../library/bibliography.bib}

\title{Meetrapport: Opamps - Resultaten}
\author{Robin Hes\\\&\\Erwin R. de Haan}

\begin{document}
\chapter{Resultaten}
\section{Metingen}

We vullen de gebruikte componentwaarden in vergelijking \ref{eq:freqOscillator}:

$$f(25000) = \frac{1}{4\cdot 25000C \cdot 10 \times 10^{-9}} = 1000 \mathrm{Hz}$$

\noindent
De gemeten frequentie met de potmeter op de maximale waarde $f_{R_{pot,max}} = 1070 \mathrm{Hz}$. Het verschil tussen deze waarden (70 Hz, $\frac{70}{1000} \cdot 100 = 7\%$) is een verschil dat onder andere te wijten is aan afwijkingen in de gebruikte componenten (weerstandswaarden, interne weerstand van op-amps enz.) en de gebruikelijke toevallige meetfouten. Deze waarden stemmen dus overeen en zijn voor onze doeleinden voldoende nauwkeurig.

\begin{table}[H]
	\caption{De componentwaarden in de relaxatie-oscillator.}
	\label{tab:components}
	\centering
	\begin{tabular}{|c|c|}
		\hline
		Naam & Waarde\\
		\hline
		$R_a$ & $1 \mathrm{k\Omega}$\\
		\hline
		$R_b$ & $10 \mathrm{k\Omega}$\\
		\hline
		$R_c$ & $10 \mathrm{k\Omega}$\\
		\hline
		$R_d$ & $1 \mathrm{k\Omega}$\\
		\hline
		$R_{pot}$ & $0\--25 \mathrm{k\Omega}$\\
		\hline
		$C$ & $10 \mathrm{nF}$\\
		\hline
	\end{tabular}
\end{table}

\section{Mogelijke implementatie}

In de FPGA kunnen we het aantal klokpulsen tellen tussen twee events die het signaal hoog maken. We zouden dit zelfs kunnen middelen met het voorgaande resultaat.

\begin{equation}
	\label{eq:fpgaEq}
	R = \frac{N_{clk}}{4Cf_{FPGA}}
\end{equation}

\noindent
Waarbij $f_{FPGA} = 50 \mathrm{MHz}$ en $N_{clk}$ is het aantal getelde klokcycli. Omdat de FPGA veel makkelijker rekent met integer waarden, kunnen we de vergelijking nog maal $10^8$ doen om $C$ ook als integer te gebruiken.

\end{document}