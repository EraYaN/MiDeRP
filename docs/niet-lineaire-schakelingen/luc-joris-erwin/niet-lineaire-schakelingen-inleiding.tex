\documentclass{report}
% Include all project wide packages here.
\usepackage{fullpage}
\usepackage[style=ieee]{biblatex}
\usepackage[dutch]{babel}
\usepackage[T1]{fontenc}
\usepackage{titlesec, blindtext, color}
\definecolor{gray75}{gray}{0.75}
\newcommand{\hsp}{\hspace{20pt}}
\titleformat{\chapter}[hang]{\Huge\bfseries}{\thechapter\hsp\textcolor{gray75}{|}\hsp}{0pt}{\Huge\bfseries}
\renewcommand{\familydefault}{\sfdefault}
\usepackage[math]{iwona}

\addbibresource{../../library/bibliography.bib}

\title{EPO-2: Niet-lineaire schakelingen - Inleiding}
\author{Luc Does}

\begin{document}

\chapter{Inleiding}
\label{ch:inleiding}

In dit verslag behandelen wij de spanningsgestuurde spanningsdeler m.b.v. een diode als voorbeeld van het gebruik van niet lineaire schakelingen. Deze opdracht sluit aan bij het EPO-2 project en heeft onder meer als doel het gebruik van simulaties te verduidelijken bij de bouw van niet-lineaire schakelingen. \newline
Aangezien de groep is verdeelt in twee delen behandelen wij hier de eerste opdracht om deze uit te leggen aan de andere groep; welke de tweede opdracht heeft behandelt. Bij deze uitleg geven wij de uitwerkingen van de opgaves op een verklarende manier.


\section{1}
Om de diode als een spannings-gestuurde spanningsdeler te gebruiken is het belangrijk om een juiste keuze voor de waardes van $R$ en $V_C$ te maken. Aangezien het analytisch oplossen van niet-lineaire schakelingen zeer lastig is is het gebruik van een simulatieprogramma ideaal om een goede benadering te krijgen van de benodigde waardes. In plaats van de aangewezen Falstad simulator hebben wij gebruik gemaakt van de CircuitLab simulator. Deze simulator is vriendelijker in gebruik terwijl hij wel op het geavanceerdere  Spice is gebaseerd, hierdoor is het voor ons makkelijk om nauwkeurige simulaties uit te voeren.
\newline
De karakteristiek die wij zoeken bij een goed functionerende schakeling is ... Bij de waardes $R = $ en $V_C = $ zien we deze karakteristiek terug en daarom hebben we deze gekozen als eerste, ruwe waardes.

\section{2}
Bij een groot-signaal circuit gaan we uit van spanningen binnen de schakeling die vele malen groter zijn dan de klein-signalen die de sensor produceert. Hierdoor valt dit deel van de schakeling weg, het circuit dat resteert wordt gegeven door \ref{fig:groot}.
We zien dat de wisselspanningsbron is weggevallen aangezien deze een kleinsignaal inhoudt, deze spanningsbron is daarbij een kortsluiting geworden. Ook zien we dat de condensatoren zijn omgevormd tot open klemmen, dit is een gevolg van de afwezigheid van een wisselspanning.
\begin{figure}[H]
	\centering
	\includegraphics[width=0.8\textwidth]{grootsignaal-model.pdf}
	\caption{grootsignaal-model van de schakeling}
	\label{fig:groot}
\end{figure}

Als we deze schakeling simuleren komen we erachter dat het instelpunt ligt op ...

\section{3}
Met behulp van het CircuitLab voeren wij een DC-sweep uit voor het circuit. Door de spanning over de diode te delen door de stroom die erdoorheen loopt kunnen wij de differentiële weerstand bepalen. Deze differentiële weerstand staat geplot in figuur \ref{fig:rdv}
\begin{figure}[H]
	\centering
	\includegraphics[width=0.8\textwidth]{iv.png}
	\caption{$iv$-karakteristiek van de schakeling}
	\label{fig:iv}
\end{figure}

\begin{figure}[H]
	\centering
	\includegraphics[width=0.8\textwidth]{RdV.png}
	\caption{$r_dv$-karakteristiek van de schakeling}
	\label{fig:rdv}
\end{figure}


\end{document}