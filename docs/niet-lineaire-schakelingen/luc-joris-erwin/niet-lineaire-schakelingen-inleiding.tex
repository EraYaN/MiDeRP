\documentclass{report}
% Include all project wide packages here.
\usepackage{fullpage}
\usepackage[style=ieee]{biblatex}
\usepackage[dutch]{babel}
\usepackage[T1]{fontenc}
\usepackage{titlesec, blindtext, color}
\definecolor{gray75}{gray}{0.75}
\newcommand{\hsp}{\hspace{20pt}}
\titleformat{\chapter}[hang]{\Huge\bfseries}{\thechapter\hsp\textcolor{gray75}{|}\hsp}{0pt}{\Huge\bfseries}
\renewcommand{\familydefault}{\sfdefault}
\usepackage[math]{iwona}

\addbibresource{../../library/bibliography.bib}

\title{EPO-2: Niet-lineaire schakelingen - Inleiding}
\author{Luc Does}

\begin{document}

\chapter{Inleiding}
\label{ch:inleiding}

In dit verslag behandelen wij de spanningsgestuurde spanningsdeler m.b.v. een diode als voorbeeld van het gebruik van niet lineaire schakelingen. Deze opdracht sluit aan bij het EPO-2 project en heeft onder meer als doel het gebruik van simulaties te verduidelijken bij de bouw van niet-lineaire schakelingen. \newline
Aangezien de groep is verdeelt in twee delen behandelen wij hier de eerste opdracht om deze uit te leggen aan de andere groep; welke de tweede opdracht heeft behandelt. Bij deze uitleg geven wij de uitwerkingen van de opgaves op een verklarende manier.

\end{document}