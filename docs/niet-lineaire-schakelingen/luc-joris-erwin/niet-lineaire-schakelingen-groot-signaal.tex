\documentclass{report}
% Include all project wide packages here.
\usepackage{fullpage}
\usepackage[style=ieee]{biblatex}
\usepackage[dutch]{babel}
\usepackage[T1]{fontenc}
\usepackage{titlesec, blindtext, color}
\definecolor{gray75}{gray}{0.75}
\newcommand{\hsp}{\hspace{20pt}}
\titleformat{\chapter}[hang]{\Huge\bfseries}{\thechapter\hsp\textcolor{gray75}{|}\hsp}{0pt}{\Huge\bfseries}
\renewcommand{\familydefault}{\sfdefault}
\usepackage[math]{iwona}

\addbibresource{../../library/bibliography.bib}

\title{EPO-2: Niet-lineaire schakelingen - Groot signaal equivalent}
\author{Luc Does}

\begin{document}

\chapter{Het groot signaal equivalent}
\label{ch:Groot signaal equivalent}

Bij een groot-signaal circuit gaan we uit van spanningen binnen de schakeling die vele malen groter zijn dan de klein-signalen die de sensor produceert. Hierdoor is dit deel van de schakeling te verwaarlozen, het circuit dat resteert wordt gegeven door \ref{fig:groot}.
We zien dat de wisselspanningsbron is weggevallen aangezien deze een klein-signaal inhoudt, deze spanningsbron is daarbij een kortsluiting geworden. Ook zien we dat de condensatoren zijn omgevormd tot open klemmen, dit is een gevolg van de afwezigheid van een wisselspanning.
\begin{figure}[H]
	\centering
	\includegraphics[width=0.3\textwidth]{grootsignaalmodel.png}
	\caption{grootsignaal-model van de schakeling}
	\label{fig:groot}
\end{figure}

\noindent Wanneer we een DC-sweep uitvoeren voor Vc met CircuitLab met betrekking op de stroom door de diode krijgen we de grafiek uit figuur \ref{fig:iv-groot}. Hier valt op te merken dat tot ongeveer 1V een non-lineair verband bestaat tussen de stroom door de diode en de instelspanning. Bij 1.1V kan de spanning verhoogt of verlaagt worden zonder dat het systeem zijn lineariteit verliest. Vandaar dat we 1.1V als instelpunt hebben gebruikt. Uit de simulatie is ook verkregen dat bij dit instelpunt van 1.1V er een spanning van 839.0mV staat over de diode.

\begin{figure}[H]
	\centering
	\includegraphics[width=1.1\textwidth]{iv-groot.png}
	\caption{$iv$-karakteristiek van de grootsignaal schakeling}
	\label{fig:iv-groot}
\end{figure}

\end{document}