\documentclass{report}
% Include all project wide packages here.
\usepackage{fullpage}
\usepackage[style=ieee]{biblatex}
\usepackage[dutch]{babel}
\usepackage[T1]{fontenc}
\usepackage{titlesec, blindtext, color}
\definecolor{gray75}{gray}{0.75}
\newcommand{\hsp}{\hspace{20pt}}
\titleformat{\chapter}[hang]{\Huge\bfseries}{\thechapter\hsp\textcolor{gray75}{|}\hsp}{0pt}{\Huge\bfseries}
\renewcommand{\familydefault}{\sfdefault}
\usepackage[math]{iwona}

\addbibresource{../../library/bibliography.bib}

\title{EPO-2: Niet-lineaire schakelingen - Conclusie}
\author{Joris Blom}

\begin{document}

\chapter{Conclusie}
\label{ch:conclusie}
Het werken met een spanningsgestuurde spanningsdeler met behulp van een diode geeft ons inzicht in de werking van niet-lineaire elektrische componenten.

In het klein-signaal model is goed de werking van het circuit te zien. Met dit circuit kunnen we doormiddel van een regelbare gelijkspanning de weerstand van de diode regelen, waardoor de ingangsspanning op een andere manier gedeeld wordt.
\end{document}