\documentclass{report}
% Include all project wide packages here.
\usepackage{fullpage}
\usepackage[style=ieee]{biblatex}
\usepackage[dutch]{babel}
\usepackage[T1]{fontenc}
\usepackage{titlesec, blindtext, color}
\definecolor{gray75}{gray}{0.75}
\newcommand{\hsp}{\hspace{20pt}}
\titleformat{\chapter}[hang]{\Huge\bfseries}{\thechapter\hsp\textcolor{gray75}{|}\hsp}{0pt}{\Huge\bfseries}
\renewcommand{\familydefault}{\sfdefault}
\usepackage[math]{iwona}

\addbibresource{../../library/bibliography.bib}

\title{EPO-2: Niet-lineaire schakelingen - Waarden voor $R_c$ en Vc bepalen}
\author{Joris Blom}

\begin{document}

\chapter{Waarden voor $R_c$ en Vc bepalen}
\label{ch:Waarden voor $R_c$ en Vc bepalen}
Voor het bepalen van de equivalente weerstand te bepalen voor weerstanden die parallel staan wordt de deze formule gebruikt:
\begin{equation}
\label{eq:resR}
R' = R_1||R_2||R_3... = \frac{1}{\frac{1}{R_1}+\frac{1}{R_2}+\frac{1}{R_3}...}
\end{equation}

\noindent Voor het bepalen van de spanning tussen 2 weerstanden die in serie staan wordt deze formule gebruikt:
\begin{equation}
\label{eq:overdracht}
V_o =V_i\cdot \frac{R_1}{R_1+R_2}
\end{equation}

\section{Het klein-signaal model}
\label{ch:Het klein-signaal model}
\begin{figure}[h]
\centering
\includegraphics[width=\textwidth]{kleinsignaalmodel.png}

\caption{klein-signaal model}
\end{figure}
Hierboven is het klein-signaal model weergeven van de spanningsgestuurde spanningsdeler. Als de waarde van $R_c$ te klein is dan zal er een grotere wisselspanning staan over de $100\Omega$ weerstand en dus is de spanning aan de uitgang lager. Hierdoor zal de maximale uitgangsspanning van de spanningsgestuurde spanningsdeler ook kleiner worden. Als de waarde van $R_c$ te groot is dan zal er een kleinere wisselspanning over de $100\Omega$ weerstand staan, waardoor de maximale uitgangsspanning groter is. Er is gekozen om een $1k\Omega$ weerstand te gebruiken voor $R_c$. Met behulp van vergelijking \ref{eq:resR} en \ref{eq:overdracht} is de maximale uitgangsspanning bepaald.


$$
\label{eq:maximaalVoltage}
V_{max} = V_i * \frac{R_c||R_l}{R_c||R_l + 100} = 4.32 mV
$$

Een 1N4007 diode kan maximaal 1 watt aan vermogen dissiperen. Als Vc boven de 1000 volt komt, is het vermogen dat de diode dissipeert ongeveer 1 watt. Spanningen van 0 tot 1000 volt zijn dus bruikbaar. Hierbij is uitgegaan dat er ideale weerstanden, condensatoren en spanningsbronnen gebruikt worden.



\end{document}