\documentclass{report}
% Include all project wide packages here.
\usepackage{fullpage}
\usepackage[style=ieee]{biblatex}
\usepackage[dutch]{babel}
\usepackage[T1]{fontenc}
\usepackage{titlesec, blindtext, color}
\definecolor{gray75}{gray}{0.75}
\newcommand{\hsp}{\hspace{20pt}}
\titleformat{\chapter}[hang]{\Huge\bfseries}{\thechapter\hsp\textcolor{gray75}{|}\hsp}{0pt}{\Huge\bfseries}
\renewcommand{\familydefault}{\sfdefault}
\usepackage[math]{iwona}

\addbibresource{../../library/bibliography.bib}

\title{EPO-2: Niet-lineaire schakelingen - Eerste simulatie van het circuit}
\author{Luc Does}

\begin{document}

\chapter{Eerste simulatie van het circuit}
\label{ch:Eerste simulatie van het circuit}

Om de diode als een spannings-gestuurde spanningsdeler te gebruiken is het belangrijk om een juiste keuze voor de waardes van $R1$ en $V1$ te maken. Aangezien het analytisch oplossen van niet-lineaire schakelingen zeer lastig is is het gebruik van een simulatieprogramma ideaal om een goede benadering te krijgen van de benodigde waardes. In plaats van de aangewezen Falstad simulator hebben wij gebruik gemaakt van de CircuitLab simulator. Deze simulator is vriendelijker in gebruik terwijl hij wel op het geavanceerdere  Spice is gebaseerd, hierdoor is het voor ons makkelijk om nauwkeurige simulaties uit te voeren.
\newline
De karakteristiek die wij zoeken bij een goed functionerende schakeling is een lineair verband tussen de stroom en de spanning. Bij de waardes $R1 = 1k\Omega$ en $V1 = 1V$ zien we deze karakteristiek terug en daarom hebben we deze gekozen als eerste ruwe waardes.

\end{document}