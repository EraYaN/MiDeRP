\documentclass{report}
% Include all project wide packages here.
\usepackage{fullpage}
\usepackage[style=ieee]{biblatex}
\usepackage[dutch]{babel}
\usepackage[T1]{fontenc}
\usepackage{titlesec, blindtext, color}
\definecolor{gray75}{gray}{0.75}
\newcommand{\hsp}{\hspace{20pt}}
\titleformat{\chapter}[hang]{\Huge\bfseries}{\thechapter\hsp\textcolor{gray75}{|}\hsp}{0pt}{\Huge\bfseries}
\renewcommand{\familydefault}{\sfdefault}
\usepackage[math]{iwona}

\addbibresource{../../library/bibliography.bib}

\begin{document}

\section{Weerstandswaarden $R_1$, $R_2$}

\begin{equation}
\frac{V_{CC}-V_{BE}}{R_1}=I_B 
\end{equation}
$$\Rightarrow R_1=\frac{V_{CC}-V_{BE}}{I_B}$$

\noindent De basisstroom wordt berekend door de collectorstroom te delen met de versterkingsfactor van de transistor,

\begin{equation}
I_C=I_B \cdot \beta
\end{equation}
$$\Rightarrow I_B=\frac{I_C}{\beta}$$
$$\Rightarrow I_B=\frac{10 \: mA}{75}=1.3\times 10^{-4}A$$

\noindent Alle waarden in formule 1 invullen geeft,
$$\Rightarrow R_1=\frac{9\: V-0.7\: V}{1.3\times 10^{-4}A}$$
$$\Rightarrow R_1=63.8 \: k  \Omega$$

\noindent Om de waarden van $R_2$ te berekenen, veronderstellen we dat de collector spanning $V_C$ gelijk is aan 4.5 V. Een spanning van 4.5 V geeft het meeste bereik naar de aarde en naar de voedingsspanning $V_{CC}$ van 9 V. 

\begin{equation}
V = I \cdot R 
\end{equation}

$$R_2 = \frac{V_{CC}-V_C}{I_C} = \frac{9\:V-4.5\:V}{10\:mA}=450\: \Omega$$




\end{document}