\documentclass[11pt]{article}

\begin{document}
\section{Groot-signaal model en instelpunt}

Het groot-signaal equivalent circuit maakt het mogelijk om de niet-lineaire schakeling te beschrijven. De spanning die wordt aangeboden aan de ingang is een sinusvormig signaal en kan al bij grotere waarden ertoe leiden dat de uitgangsspanning geen lineaire versterking is van de ingangsspanning.

De versterkingsfactor van de 2N2222 transistor is 75. Als er een bepaalde ingangsspanning wordt geleverd, kan het zo zijn dat de voedingsspanning van 9 V de versterking niet kan leveren. Hierdoor ontstaat er niet een sinusgrafiek met een grotere amplitude, maar bijvoorbeeld een blokgolf (de uitgangsspanning wordt geclipt).\\

Het instelpunt Q is maximale waarde van de uitgangsspanning waarbij de sinusgrafiek stabiel is. Dat wil zeggen: het punt waarbij de uitgangsspanning net niet clipt. Met het groot-signaal model kan het instelpunt worden bepaald.

\begin{equation}
I_B=\frac{V_{CC}-V_{BE}}{R_B}
\end{equation}
$$\Rightarrow I_B=\frac{9\: V-0.7 \: V}{64 \: k \Omega}=1.3\times 10^{-4}A$$

De collector stroom $I_C$ wordt gegeven door, 
\begin{equation}
I_C=\beta I_B
\end{equation}
$$\Rightarrow I_C=75 \times 1.3\times 10^{-4}A=9.7 \: mA$$
De uitgangsspanning $V_o$ is de voedingsspanning minus de spanningsval over $R_2$ ($R_C$),
\begin{equation}
V_o=V_{CC}-I_{C}R_{C}
\end{equation}
$$\Rightarrow V_o=9-9.7\times 10^{-3}\times 875\approx 0.49 \: V$$

\noindent Dus, het instelpunt van de schakeling is bij een collector stroom van 9.7 mA en een uitgangsspanning van ongeveer 0.49 V.

\newpage
\section{Klein-signaal model en versterkingsfactor}

De spanningsversterking is het klein-signaal model wordt gegeven door,

\begin{equation}
A_v=\frac{v_o}{v_i}
\end{equation}
$$v_o=-g_m\cdot u_{be}\cdot R_2'$$
Hierin is $R_2'$ de vervangende waarde voor de parallelschakeling van $R_2$ en $r_{ce}$,
$$R_2'=\frac{R_2\cdot r_{ce}}{R_2+r_{ce}}$$

\noindent Het is meestal zo dat $r_{ce}\gg R_2$, hierdoor wordt $R_2'\approx R_2$,
$$\Rightarrow v_o=-g_m\cdot u_{be}\cdot R_2$$

\noindent De waarde van $g_m$ kan berekend worden door middel van,

\begin{equation}
g_m=\frac{I_C}{V_T}
\end{equation}
Hierin is $I_C$ de collector stroom bij het instelpunt en $V_T$ de thermische spanning van de transistor op kamertemperatuur, de waarde is ongeveer 25 mV.

$$\Rightarrow g_m=\frac{9.7\times 10^{-3}A}{25\times 10^{-3}V}= 0.39$$
\end{document}