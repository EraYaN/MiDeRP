\documentclass[dutch]{article}
\usepackage{amsmath}
\usepackage{mathenv}
\usepackage{babel} 
\usepackage{verbatim}

\begin{document}

\title{Notulen EPO-2 D-2 week 3.2}
\author{L.T. Does}

\maketitle
\noindent
Vergadering geopend 15:05 \newline
Locatie: Drebbelweg zaal 1.060".\newline
\newline
Aanwezigen
\begin{itemize}
\item Tijmen Witte (voorzitter)
\item Luc Does (notulist)
\item Erwin de Haan
\item Robin Hes
\item Chy Lau
\item Joris Blom
\item Sander de Graaf (tutor)
\end{itemize}

\begin{enumerate}

\item Opening
\begin{itemize}
\item Tijmen opent de vergadering.
\end{itemize}
\item Notulen vorige vergadering
\begin{itemize}
\item EPO-2 moet worden geschreven met hoofdletters.
\item Agenda en notulen moeten tenminste één dag voor de vergadering aanwezig zijn.
\end{itemize}
\item Just in time training; Informatievaardigheden
\begin{itemize}
\item Alle projectleden hebben hun just in time training afgerond.
\end{itemize}
\item Plan van aanpak
\begin{itemize}
\item Bij het plan van aanpak hoeven alleen de belangrijkste aspecten van de methoden uit 'Projectmanagement' van Grit toegepast te worden.
\end{itemize}
\item Wat verder ter tafel komt
\begin{itemize}
\item Deadlines mogen in de gespecificeerde week nog ingeleverd worden, toch liever voor aanvang van de eerste projectdag in deze week.
\end{itemize}
\item Samenvatting besluiten
\item Samenvatting actiepunten
\begin{itemize}
\item Elk subgroepje moet zijn C-code inleveren, hierbij moeten zij een Readme-file samenvoegen waarin wordt uitgelegd hoe de code functioneert en gebruikt dient te worden.
\item Er wordt een rooster gemaakt om de rollen van voorzitter en notulist te rouleren.
\end{itemize}
\end{enumerate}

\end{document}

